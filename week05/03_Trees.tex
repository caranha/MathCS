\section{Trees}

\frame{
{Part 3: Trees}

\tableofcontents[currentsection,hideallsubsections, firstsection=1, sections={1-4}]
}


\begin{frame}{Trees and Connectivity}

  {\larger
    \begin{itemize}
    \item \structure{Trees} are connected Graphs with \alert{no cycles}.
    \item Every tree 1-edge connectivity, 1-vertex connectivity.
    \item Chromatic Number = 2 (trees can always be bi-colored)

      \bigskip

    \item Trees come up all the time:
      \begin{columns}
        \column{0.5\textwidth}
        \begin{itemize}
        \item Family Trees;
        \item Search Trees;
        \item Game Trees;
        \item Parse Trees;
        \end{itemize}
        \column{0.5\textwidth}
        \begin{itemize}
        \item Spanning Trees;
        \item Rooted Trees;
        \item Ordered Trees;
        \item Binary Trees;
        \item etc...
        \end{itemize}
      \end{columns}
    \end{itemize}
  }
\end{frame}

\begin{frame}{Trees and Connectivity}

  {\larger
    \begin{itemize}
    \item \structure{Cut Edge}: An edge is a cut edge if
      removing it makes two vertices disconnected.

      \bigskip

    \item {\bf Lemma:} An edge is not a cut edge if it is on a
      cycle.

      \bigskip

    \item A tree is a \structure{connected graph} where
      \structure{every edge is a cut edge}

      \bigskip

    \item This implies that a tree is a connected graph which
      is {\bf\structure{Edge Minimal}}
      \begin{itemize}
      \item A tree has the minimum number of edges necessary to
        connect a set of vertices.
      \end{itemize}
    \end{itemize}
  }
\end{frame}

\begin{frame}
  \frametitle{Tree Coloring}

  {\larger
    \begin{itemize}
    \item A tree is a graph with a \structure{unique path}
      between every pair of vertices.

    \item As a consequence, $\chi(\text{tree}) = 2$

    \item {\bf Constructive Demonstration}

      \begin{columns}
        \column{0.7\textwidth}
        \begin{itemize}
        \item Pick any node in the tree to be the {\bf root}, color
          it ``blue''.
        \item Color nodes ``odd'' length from the root as ``red''
        \item Color nodes ``even'' length from the root as ``blue''
        \item This is the algorithm for 2-coloring on general graphs
        \end{itemize}
        \column{0.3\textwidth}

        \begin{tikzpicture}[scale=.8,auto,swap]
          %\tikzset{edge/.style = {->,>=latex'}}
          \node[blue vertex] (a) at (1,0) {};
          \node[red vertex] (a1) at (0,1) {};
          \node[red vertex] (b1) at (1,1) {};
          \node[blue vertex] (a2) at (0,2) {};
          \node[blue vertex] (b2) at (1,2) {};
          \node[blue vertex] (c2) at (2,2) {};
          \node[red vertex] (a3) at (0,3) {};
          \node[red vertex] (b3) at (1,3) {};
          \node[red vertex] (c3) at (2,3) {};
          \node[red vertex] (d3) at (3,3) {};
          \draw[edge] (a) to (a1);
          \draw[edge] (a) to (b1);
          \draw[edge] (a1) to (a2);
          \draw[edge] (a1) to (b2);
          \draw[edge] (b1) to (c2);
          \draw[edge] (b2) to (a3);
          \draw[edge] (b2) to (b3);
          \draw[edge] (b2) to (c3);
          \draw[edge] (c2) to (d3);

        \end{tikzpicture}
      \end{columns}

    \end{itemize}



  }
\end{frame}

\subsection{Spanning Trees}

\begin{frame}
  \frametitle{Spanning Trees}

  {\larger

    \begin{itemize}
    \item A \structure{Spanning Subgraph} of G is a subgraph of
      $G$ that has all vertices of $G$ (and some of the edges).
    \item A \structure{Spanning Tree} of G is a spanning graph of
      $G$ that is also a tree.
    \end{itemize}

    \begin{center}
      \begin{tikzpicture}[scale=.8,auto,swap]
        %\tikzset{edge/.style = {->,>=latex'}}
        \node[vertex] (a) at (0,1) {};
        \node[vertex] (a1) at (1,0) {};
        \node[vertex] (b1) at (1,1) {};
        \node[vertex] (a2) at (2,0) {};
        \node[vertex] (b2) at (2,1) {};
        \node[vertex] (c2) at (2,2) {};
        \node[vertex] (a3) at (3,0) {};
        \node[vertex] (b3) at (3,1) {};
        \node[vertex] (c3) at (3,2) {};
        \node[vertex] (d3) at (2,3) {};
        \draw[red edge] (a) to (a1);
        \draw[red edge] (a) to (b1);
        \draw[red edge] (a1) to (a2);
        \draw[red edge] (a1) to (b2);
        \draw[red edge] (b1) to (c2);
        \draw[red edge] (b2) to (a3);
        \draw[red edge] (b2) to (b3);
        \draw[red edge] (b2) to (c3);
        \draw[edge] (c2) to (d3);
        \draw[edge] (a) to (d3);
        \draw[edge] (b1) to (d3);
        \draw[edge] (a1) to (b1);
        \draw[red edge] (b1) to (b2);
        \draw[edge] (b2) to (a2);
        \draw[edge] (b2) to (c2);
        \draw[red edge] (a2) to (a3);
        \draw[red edge] (c2) to (c3);
        \draw[edge] (a3) to (b3);
        \draw[edge] (c3) to (d3);
        \draw[edge] (b3) to (c3);
      \end{tikzpicture}
    \end{center}

    \begin{itemize}
    \item One graph can have multiple spanning trees.
    \item Every connected graph has a spanning tree.
    \end{itemize}
  }
\end{frame}

\begin{frame}
  \frametitle{Weighted Spanning Trees}

  {\larger

    The Spanning Tree problem becomes more interesting when we consider
    \structure{weighted edges}.

    \begin{center}
      \begin{tikzpicture}[scale=.8,auto,swap]
        %\tikzset{edge/.style = {->,>=latex'}}
        \node[vertex] (a) at (3,5) {};
        \node[vertex] (b) at (0,3) {};
        \node[vertex] (c) at (3,3) {};
        \node[vertex] (d) at (6,3) {};
        \node[vertex] (e) at (2,1) {};
        \node[vertex] (f) at (4,1) {};
        \draw[edge] (a) -- node[weight] {$3$} (b);
        \draw[edge] (a) -- node[weight] {$4$} (d);
        \draw[edge] (b) -- node[weight] {$4$} (c);
        \draw[edge] (c) -- node[weight] {$1$} (d);
        \draw[edge] (b) -- node[weight] {$6$} (e);
        \draw[edge] (c) -- node[weight] {$9$} (e);
        \draw[edge] (c) -- node[weight] {$7$} (f);
        \draw[edge] (d) -- node[weight] {$2$} (f);
        \draw[edge] (e) -- node[weight] {$1$} (f);
      \end{tikzpicture}
    \end{center}

    What is the \structure{minimal cost} structure that
    allows me to connect everything?

  }
\end{frame}

\begin{frame}
  \frametitle{Minimum Spanning Tree Algorithm}

  {\larger
    \begin{center}
      \begin{tikzpicture}[scale=.8,auto,swap]
        %\tikzset{edge/.style = {->,>=latex'}}
        \node[vertex] (a) at (3,5) {};
        \node[vertex] (b) at (0,3) {};
        \node[vertex] (c) at (3,3) {};
        \node[vertex] (d) at (6,3) {};
        \node[vertex] (e) at (2,1) {};
        \node[vertex] (f) at (4,1) {};
        \draw[black edge] (a) -- node[weight] {$3$} (b);
        \draw[edge] (a) -- node[weight] {$4$} (d);
        \draw[black edge] (b) -- node[weight] {$4$} (c);
        \draw[black edge] (c) -- node[weight] {$1$} (d);
        \draw[edge] (b) -- node[weight] {$6$} (e);
        \draw[edge] (c) -- node[weight] {$9$} (e);
        \draw[edge] (c) -- node[weight] {$7$} (f);
        \draw[black edge] (d) -- node[weight] {$2$} (f);
        \draw[black edge] (e) -- node[weight] {$1$} (f);
      \end{tikzpicture}
    \end{center}

    \begin{enumerate}
    \item Start with one arbitrary vertex and add it to the MST.
    \item From all edges connected with the MST, select one with minimum weight;
    \item Add the edge, and vertex, to the MST;
    \item Return to (2)
    \end{enumerate}
  }
\end{frame}
