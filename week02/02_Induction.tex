\section{Induction}

\frame{
{Part 2: Induction}

\tableofcontents[currentsection,hideallsubsections, firstsection=2, sections={2-4}]
}

\subsection{Introductory Examples}



\begin{frame}{An initial induction (1/2)}
  Imagine that I want to color the natural numbers ($\mathbb{N} \geq 0$), using the following rules:\bigskip

  \begin{itemize}
    \item Number $0$ is \alert{red}
    \item Any integer next to a \alert{red} number is also \alert{red}
  \end{itemize}\bigskip

  Using these rules, can you imagine how the set $\mathbb{N}$ looks like?
\end{frame}

\begin{frame}{An initial induction (2/2)}

  \begin{center}
    Result: \alert{0,1,2,3,4,...}
  \end{center}\bigskip

  The "rule of reds" gives us a general idea of induction:

  \begin{itemize}
  \item $R(0)$ is True
  \item $R(0) \rightarrow R(1); R(1) \rightarrow R(2); R(2) \rightarrow R(3); \ldots$
  \item $R(n) \rightarrow R(n+1)$ for every $n \in \mathbb{N}$
  \end{itemize}\bigskip

  Induction can be used to prove a predicate that depends on some $n \in \mathbb{N}$ by modus ponens.
  \begin{equation*}
    \frac{R(0), R(n)\rightarrow R(n+1), n\in\mathbb{N}}{\forall n, R(n)}
  \end{equation*}
\end{frame}

\begin{frame}{Example of proof by Induction}

  Let's prove that:
  \begin{equation*}
    P(n): 1 + r + r^2 + r^3 + \ldots + r^n = \frac{r^{n+1}-1}{r-1}, r \neq 1, \forall n \in \mathbb{N}
  \end{equation*}\bigskip

  Remember the modus ponens rule for induction:
  \begin{equation*}
    \frac{P(0), P(n)\rightarrow P(n+1), n\in\mathbb{N}}{\forall n, P(n)}
  \end{equation*}\bigskip

  To prove the bottom part by induction, we need to prove the top part.

  \begin{itemize}
  \item First Step: Prove $P(0)$
  \item Second Step: Prove $P(n) \rightarrow P(n+1)$
  \end{itemize}
\end{frame}

\begin{frame}{Example of proof by Induction}

  \begin{proof}
    Proof by induction on $n$

    {\bf First Step:} Prove $P(0)$
    \begin{itemize}
    \item $P(0)$, left side: $r^0 = 1$
    \item $P(0)$, right side: $\frac{r^{0+1}-1}{r-1} = \frac{r-1}{r-1} = 1$
    \end{itemize}\medskip

    {\bf Second Step:} Prove $P(n) \rightarrow P(n+1)$
    \begin{itemize}
    \item $P(n+1)$, left side: $1 + r + r^2 + \ldots + r^n + r^{n+1}$, which is equal to $P(n)+r^{n+1}$
    \item Because $P(n)$ is True, $P(n)+r^{n+1} = \frac{r^{n+1}-1}{r-1} + r^{n+1} = \frac{(r^{n+1}-1)}{r-1} + \frac{(r^{n+1}(r-1))}{r-1}$
    \item Algebra: $\frac{(r^{n+1}-1) + (r^{n+1}(r-1))}{r-1} = \frac{r^{n+1} - 1 + r^{n+2} - r^{n+1}}{r-1} = \frac{r^{n+2} - 1 + (r^{n+1} - r^{n+1})}{r-1}$
    \item $\frac{r^{n+2} - 1}{r-1} = \frac{r^{(n+1)+1} - 1}{r-1}$, which is the right side of $P(n+1)$
    \end{itemize}
  \end{proof}
\end{frame}

\subsection{Proof by Induction}
\begin{frame}{Review: Proof Template for Induction}

  {\larger
    {\bf Proof by induction on $n$}\\
    Proof hypothesis: $P(n) = \ldots$ for all $n \in \mathbb{N}. n \geq 0$\\

    \bigskip

    First we prove $P(0)$.\\
    $\ldots$ \emph{(calculate that P(0) is True)}\\
    $\ldots$\\

    \bigskip

    Second we prove that $\forall n \geq 0, P(n) \rightarrow P(n+1)$\\
    $\ldots$ \emph{(calculate P(n+1) using P(n))}\\
    $\ldots$\\

    \bigskip

    This completes the proof that $P(n)$ for all $n\in\mathbb{N}$\hfill$\qed$
  }
\end{frame}

\subsection{The Statue Park}

\begin{frame}{The Statue Park}{A more complex proof by induction}

  The university is making a new park with the following rules:
  \begin{itemize}
    \item The park is square, with side $2^n$;
    \item In the middle of the park, there is a statue, size $1\times 1$;
    \item Other than that, the park is made of L-shaped tiles, with size $3m^2$;
  \end{itemize}\bigskip

  How can we prove that it is possible to build this park for any $n$?
\end{frame}

\begin{frame}[t]{The Statue Park}{Drawing Proof}
  Remember the rule of induction:
  \begin{itemize}
    \item Prove that $P(0)$ is true.
    \item Assume that $P(n)$ is true, then prove that $P(n) \implies P(n+1)$
  \end{itemize}
\end{frame}


\subsection{BAD induction proof}

\begin{frame}[t]{BAD Proof by induction: All horses are of the same color}

  $P(n) ::=$ for any \structure{set} with \alert{exactly n horses}, all horses have the same color.\bigskip

  \begin{itemize}

  \item<2-> {\bf Prove P(1)}: For any set with one horse, all horses have the same (one) color.

  \item<3-> {\bf Assume P(n) is true:} For any set with $n$ horses, all horses have the same color.
  \item<4-> {\bf Show that $P(n) \implies P(n+1)$:}
    \begin{itemize}
    \item <5->Consider the set of n+1 horses: $H = h_1, h_2, \ldots, h_n, h_{n+1}$
    \item <6->Subset A ($h_1, h_2, \ldots, h_n$): has $n$ horses, so all horses have the same color.
    \item <7->Subset B ($h_2, \ldots, h_n, h_{n+1}$): \alert{also} has $n$ horses, so all horses have the same color.
    \item <8->Horse $h_2$ is in subset $A$ {\bf and} in subset $B$, so subset $A$ and $B$ have the same color.
    \end{itemize}
  \item<9-> Since we showed that P(n+1) is true if P(n) is true, then all horses for any group size have the same color.
  \end{itemize}\bigskip

  \only<10>{\alert{QUIZ}: What is wrong with this proof?}
\end{frame}

\begin{frame}{What is wrong with the horse proof?}

  The second step, when we show that $P(n) \implies P(n+1)$ is not valid.\bigskip

  \begin{itemize}
    \item The implication proof depends on "$h_i$ belongs to subsets $A$ and $B$".
    \item But is this ALWAYS true?
    \begin{itemize}
      \item When $n+1 = 2$, The $n+1$ set is $\{h_1, h_2\}$, set $A = h_1$, set $B = h_2$;
      \item But in this case, {\bf there is no $h_i$ that is common to $A$ and $B$}!
    \end{itemize}
    \item So the implication proof is not valid when $P(2)$.
  \end{itemize}\bigskip

  Note that this is the only problem with the proof!
\end{frame}

\subsection{Strong Induction}

\begin{frame}
  \frametitle{Strong Induction}

  {\larger
    \begin{itemize}
    \item In regular induction, you assume P(n) to show P(n+1)

      \bigskip

    \item In strong induction, you assume P(0), P(1), P(2) \ldots
      P(n), and use all of them to show P(n+1)
    \end{itemize}
  }
\end{frame}

\begin{frame}
  \frametitle{Strong Induction Example: Stacking Game}
  {\larger
  \begin{itemize}
  \item Begin with a stack of 10 blocks
  \item Divide it in two (a,b): for example, 2 and 8 blocks.
  \item You get $a\times b$ points: 10 points
  \item Repeat with the new stacks until all stacks have 1 block.
  \end{itemize}

  \bigskip

  \alert{What is the best strategy?}
  \begin{itemize}
  \item Simple strategy: 1+9, 1+8, 1+7, 1+6... \only<2->{45 points!}
  \item CS strategy: 5+5, 2+3 and 2+3, ... \only<3->{45 points!}
  \end{itemize}
  }
\end{frame}

\begin{frame}
  \frametitle{Proof: All strategies have the same score (Part I)}

  {\larger
  Let us prove by strong inductions that all strategies for the stack
  game with ``n'' blocks have the same score:
  \begin{equation*}
    C(n) = \frac{n(n-1)}{2}
  \end{equation*}

  \bigskip

  {\bf Base Cases: 0, 1}
  \begin{itemize}
  \item When the stack has 0 blocks, I have no moves, so 0 points.
  \item When the stack has 1 block, I have no moves, so 0 points.
  \end{itemize}
  \begin{equation*}
    C(0) = \frac{0(0-1)}{2}, C(1) = \frac{1(1-1)}{2} = 0
  \end{equation*}
  }
\end{frame}

\begin{frame}
  \frametitle{Proof: All strategies have the same score (Part II)}

    {\bf Inductive Case} $C(n+1)$\\
    By strong induction, we assume that all $C(0)\ldots C(n)$ are true.

    \bigskip

    \begin{enumerate}
    \item A stack with $n+1$ blocks can be split into two: $k$ and $n+1-k$\medskip

    \item The score is: $C(n+1) = k\times(n+1-k) + C(k) + C(n+1-k)$\medskip

    \item Using the strong inductive assumption: $\forall m \leq n, C(m) = \frac{m(m-1)}{2}$\medskip
    
    \item Transforming (2): $C(n+1) = \frac{2k(n+1-k)}{2} + \frac{k(k-1)}{2} +
      \frac{(n+1-k)(n-k)}{2}$
  \end{enumerate}\bigskip

  ... You can finish the calculation from here ;-)
\end{frame}
