\section{Sets and Relations}

\frame{
{Part 1: Sets and Relations}

\tableofcontents[currentsection,hideallsubsections, firstsection=2, sections={2-4}]
}

\subsection{Sets: Definitions}
\begin{frame}
  \frametitle{Mathematical Set}

  \structure{Mathematical Sets} are useful when talking about proofs. Last class we have already used some sets: \bigskip

  \begin{itemize}
    \item $\mathbb{N}$ -- Set of natural (non negative) numbers;
    \item $\mathbb{Z}$ -- Set of integer numbers;
    \item $\mathbb{R}$ -- Set of real numbers;
  \end{itemize}
\end{frame}

\begin{frame}
  \frametitle{Characteristics of sets}

  \begin{itemize}
    \item  \structure{Mathematical Sets} can mix different "types" of objects:
    \begin{itemize}
    \item \{7, ``Aranha'', $\pi/2$, TRUE\}
    \end{itemize}\bigskip

    \item \structure{Mathematical Sets} do not have a concept of "order":
    \begin{itemize}
      \item \{7, ``Aranha'', $\pi/2$, TRUE\} =   \item \{TRUE, 7, $\pi/2$, ``Aranha''\}
    \end{itemize}\bigskip

    \item \structure{Mathematical Sets} do not contain duplicates:
    \begin{itemize}
    \item \{7, $\pi$\} = \{7, $\pi$, 7\}
    \end{itemize}
  \end{itemize}
\end{frame}

\begin{frame}
  \frametitle{Set: Membership}

  The fundamental property of a set is \structure{membership}, represented by the symbol $\in$. Note that membership is \alert{not} recursive!\vfill

  \begin{columns}
    \column{.5\textwidth}
  A = \{7,TRUE,$\pi$\}
  \begin{itemize}
  \item $7 \in A$
  \item 7 is an element of $A$,
  \item $3 \notin A$
  \end{itemize}\bigskip

  \column{.5\textwidth}
  B = \{7, $\mathbb{Z}$, 3\}
  \begin{itemize}
    \item $7 \in B$
    \item $7 \in \mathbb{Z}$
    \item $\mathbb{Z} \in B$
    \item \alert{$1 \notin B$}
  \end{itemize}
\end{columns}
\end{frame}

\begin{frame}{Set: Subsets}

  {\larger
  \begin{block}{Subset}
    \begin{itemize}
    \item $A \subset B$ means that every element of A is also an element of B
    \item $A\subset B \text{ equiv }\forall x, x\in A \rightarrow x\in B$
    \item $\mathbb{Z} \subset \mathbb{R}, \mathbb{R} \subset \mathbb{C}, \{3\} \subset \{5,3,7\}$
    \end{itemize}
  \end{block}

  \begin{block}{Important!}
    \begin{itemize}
    \item $A \subset A$
    \item $\forall X\text{ is a set}, \varnothing \subset X$
    \end{itemize}
  \end{block}
  }
\end{frame}

\begin{frame}
  \frametitle{Difference between Membership and Subset}

  \begin{columns}[T]
    \column{0.5\textwidth}
    \structure{Membership} ($\in, \notin$) indicates if one member is part of a set. \structure{Subset} ($\subset, \not\subset$) indicates if one set contains the members of other sets.

    \column{0.5\textwidth}
  {\larger
  \begin{itemize}
  \item $3 \in \{3,5,6\}$
  \item $3 \not\subset \{3,5,6\}$
  \end{itemize}
  \bigskip

  \begin{itemize}
  \item $\{3\} \subset \{3,5,6\}$
  \item $\{3\} \notin \{3,5,6\}$
  \end{itemize}
  \bigskip

  \begin{itemize}
    \item $\{3\} \in \{5, 6, \{3\}\}$
  \end{itemize}
  }
  \end{columns}
\end{frame}

\begin{frame}{Power Set}
  The \structure{Power Set} of A is a special set composed of ALL subsets of A.

  \begin{equation*}
    POW(A) = \forall x \subset A, x \in POW(A)
  \end{equation*}

  For example:

  \begin{equation*}
    POW(\{T,F\}) = \{\{T\},\{F\},\{T,F\},\varnothing\}
  \end{equation*}

  The definition of power set means that for any set $A$ contained in $B$, $A$ is an element of POW(B):

  \begin{equation*}
    \mathbb{N} \in POW(\mathbb{R}), \mathbb{N} \subset \mathbb{R}, \mathbb{N} \notin \mathbb{R}
  \end{equation*}
\end{frame}


\begin{frame}{Operations on Sets}

  Finally, You should be familiar with the regular operations on sets:
  \bigskip

    \begin{itemize}
    \item Union: $A \cup B \rightarrow x \in A \lor x \in B$\bigskip
    \item Intersection: $A \cap B \rightarrow x \in A \land x \in B$\bigskip
    \item Subtraction: $A - B \rightarrow x \in A \land x \notin B$\bigskip
    \item Complement: $\overline{A} = D - A$, where $D$ is the \structure{domain}\\ (the "everything" set or "parent" set of interest).
    \end{itemize}
\end{frame}

\subsection{Sets and Proofs}

\begin{frame}{Sets, Proofs and Predicates}

  Sets are used often as parts of proofs. We used sets a little bit when we talked about the \structure{Well Ordering Principle}.\bigskip

  It is common to use predicates to determine what elements are members of a set. For example, $P(X)$ be a predicate that defines $A$. If $P(X)$ is true for a certain $X$, then $X \in A$.

  \begin{block}{Example 1}
    \begin{itemize}
    \item $A = x \in \mathbb{N}, \{x < 12 \text{ AND } x \text{ is prime}\}$
    $\implies A = \{2,3,5,7,11\}$
    \end{itemize}
    A is the set of "prime numbers smaller than 12"
  \end{block}

  \begin{block}{Example 2}
    \begin{itemize}
    \item $B = x \in \mathbb{N}, \{x \text{ is prime AND } x+2 \text{ is prime}\} \implies$\\
    $B = \{3 (5), 5 (7), 11 (13), 17 (19), 29 (31), \ldots\}$
    \end{itemize}
    B is the set of "prime numbers x where x+2 is also prime".
  \end{block}
\end{frame}


\begin{frame}{Sets and Proofs Example (1)}
{Prove that the empty set is a subset of every set.}

  \begin{proof}
    Proof by construction:

  \begin{enumerate}

  \item $A \subset B$ means that $\forall x, x \in A \rightarrow x \in B$\bigskip

  \item If $A = \varnothing$ then $x \in A$ is FALSE for $\forall x$, so we can replace ``$\forall x \in A$'' with FALSE in {\bf (1)}\bigskip

\item The statement FALSE $\rightarrow x \in B$ is always TRUE.\\ \hfill (remember that FALSE $\rightarrow X$ is always TRUE)\bigskip

  \item Therefore, $\varnothing \subset B$ is TRUE $\forall B$
  \end{enumerate}
  \end{proof}
\end{frame}


\begin{frame}{Sets and Proofs Example (2)}
{Proof that Union and Intersection are Distributive}

  \begin{equation*}
    A \cup (B \cap C) \iff (A \cup B) \cap (A \cup C)
  \end{equation*}

  \begin{proof}
    Proof by sequence of "IFF"s:
    \begin{enumerate}
    \item \structure{$x \in A \cup (B \cap C)$} {\bf iff}
    \item $x \in A \lor x \in (B \cap C)$ {\bf iff} \hfill (definition of union)
    \item $x \in A \lor (x \in B \land x \in C)$ {\bf iff} \hfill
      (definition of intersection)
    \item $(x \in A \lor x \in B) \land (x\in A \lor x \in C)$ {\bf
      iff} \hfill (distributive prop.)
    \item $(x \in A\cup B) \land (x \in A\cup C)$ {\bf iff}\hfill
      (definition of union)
    \item \structure{$x \in (A\cup B)\cap (A\cup C)$}  \hfill
      (defintion of intersection)

    \end{enumerate}
  \end{proof}
\end{frame}

\subsection{Binary Relations}
\begin{frame}
  \frametitle{Definition of Binary Relations}
  \structure{Binary Relations} define an association of elements from one set (the {\bf domain}) to another set (the {\bf co-domain}). We see (binary) relations in many different situations:\bigskip

  \begin{itemize}
    \item Functions are a special case of binary relations: f(x) = y.
    \begin{itemize}
      \item A function associates the set of inputs with the sets of outputs;
    \end{itemize}\bigskip

    \item Operations such as set membership can be expressed as binary relations.
    \begin{itemize}
      \item For example, the predicate $P(x): x is prime$ defines a binary relation from $\mathbb{N}$ to \{TRUE, FALSE\}
    \end{itemize}\bigskip

    \item "Relational Databases" (for example, SQL) are also based on the idea of binary relations
    \begin{itemize}
      \item Key of X, member of a table, joint key, etc;
    \end{itemize}
    \end{itemize}
\end{frame}

\subsection{Binary Relation Definitons}

\begin{frame}{Example of Binary Relation}

  \begin{columns}[t]
    \column{0.4\textwidth}
    Let's consider the binary relation from the set of students {\bf (D)} that are registered to the set of subjects {\bf (J)}.\bigskip

    Components of a binary relation:
    \begin{itemize}
    \item \structure{Domain}: Set of Students;
    \item \structure{Co-domain}: Set of Classes;
    \item \structure{Relation Graph}:
    \begin{itemize}
      \item Vertices: Union of Domain and Co-domain;
      \item Edges: Directed edges from Domain to Co-domain;
    \end{itemize}
    \end{itemize}\bigskip

    \column{0.6\textwidth}

    \begin{center}
      \includegraphics[width=0.6\textwidth]{../img/relations}
    \end{center}\pagenote{Relation Graph Image from MIT OCS}
    Representation of a binary relation:
    \begin{itemize}
    \item $R($Jason$) = \{6.042, 6.012\}$
    \item Jason $R$ 6.042
    \item $R(\{$Jason, Yihui$\}) = \{6.042, 6.012, 6.004\}$
    \end{itemize}
  \end{columns}
\end{frame}


\begin{frame}{Relations and Inverse Relations}

  If we think of a binary relation as a directed graph, the \structure{Inverse Relation} is the relation defined by the same graph when the edges are reversed:\bigskip

  Relation $R$:
  \begin{equation*}
    R(X) ::= {j \in J | \exists d \in X. d R j}
  \end{equation*}

  Reverse Relation $R^{-1}$:
  \begin{equation*}
    R^{-1}(Y) ::= {d \in S | \exists j \in Y. d R j}
  \end{equation*}
  \bigskip

  \begin{columns}
    \column{0.5\textwidth}
    \begin{itemize}
      \item $R(Jason) = \{6.042, 6.012\}$
      \item $R^{-1}(6.012) = \{Jason, Yihui\}$
    \end{itemize}
    \column{0.5\textwidth}
    \begin{center}
      \includegraphics[width=0.6\textwidth]{../img/relations}
    \end{center}
  \end{columns}
\end{frame}

\begin{frame}{Composite Relations}

  \begin{columns}[t]

    \column{0.5\textwidth}
    If we have a relation $V$ from set $P$ to set $D$, and a relation $R$ from set $D$ to set $J$, then we can define a \structure{composite} relation $R\circ V$ from set $P$ to set $J$ (we can also use $R(V)$).\bigskip

    \begin{itemize}
    \item $R(V(X))$ or $(R\circ V)(X)$
    \item $R(V(\text{FTL})) = \{6.003\}$
    \begin{itemize}
      \item professor FTL super{\bf V}ises Joan;\bigskip

      \item Joan is {\bf R}egistered to class 6.003;
    \end{itemize}
    \end{itemize}


    \column{0.5\textwidth}
    \begin{center}
      \includegraphics[width=0.8\textwidth]{../img/composite_relation}
    \end{center}

  \end{columns}
\end{frame}


\begin{frame}{Types of Binary Relations}

  We can classify a binary relation based on the number of degrees (arrows) in the relation graph. Imagine a relation $R$ from $X$ to $Y$ (i.e.: $R(X) = Y$)\bigskip

  Classification of $R$ based on $Y$:
  \begin{itemize}
  \item {\bf Surjection}: Every element in $Y$ has $\geq 1$ in-arrows. (Every Y has {\bf one or more} X)
  \item {\bf Injection}: Every element in $Y$ has $\leq 1$ in-arrows. (Every Y has {\bf one or less} X)
  \end{itemize}\smallskip

  Classification of $R$ based on $X$:
  \begin{itemize}
  \item {\bf Total}: Every element in $X$ has $\geq 1$ out arrows. (Every $X$ has {\bf one or more} $Y$)
  \item {\bf Function}: Every element in $X$ has $\leq 1$ out arrows. (Every $X$ has {\bf one or less $Y$})
  \end{itemize}\smallskip

  An important definition:
  \begin{itemize}
  \item {\bf Bijection}: Every element of $X$ has {\bf exactly one} element of $Y$, {\bf and vice versa}.
  \end{itemize}
\end{frame}

\begin{frame}
  \frametitle{Types of Binary Relations: Examples}

    \begin{block}{Example 1: $g: \mathbb{R}\times\mathbb{R} \rightarrow \mathbb{R}, g(x,y) = 1/(x-y)$}
      \begin{itemize}
      \item This is a {\bf function}, because each pair $(x,y)$ has at most one output.
      \item This is not {\bf total}, because not every pair $(x,y)$ has an output: $(x = y)$ has no output.
      \end{itemize}
    \end{block}

    \begin{block}{Example 2: $g_o: \mathbb{R}\times\mathbb{R} - \{x,y|x=y\} \rightarrow \mathbb{R}, g_o(x,y) = 1/(x-y)$}
      \begin{itemize}
      \item $g$ and $g_o$ are similar relations, but defined on different domains.
      \item The domain of $g_o$ removes all $(x,y)$ where $x = y$
      \item Because of this, $g_0$ is {\bf function} (every pair has at most one output) and {\bf total} (every pair has at least one output).
      \end{itemize}

    \end{block}

\end{frame}

% \begin{frame}
%   \frametitle{Size of Finite Sets}
%
%   {\larger
%     We can use the characteristics of relations to estimate the
%     size of sets (domains and co-domains).
%
%     \vfill
%
%     \begin{itemize}
%     \item A bijection B $\rightarrow |A| = |B|$
%     \item A function surjection B $\rightarrow |A| \geq |B|$
%     \item A total injection B $\rightarrow |A| \leq |B|$
%     \end{itemize}
%   }
% \end{frame}
%
% \begin{frame}
%   \frametitle{Set Size Example: Finite power sets and binary strings}
%
%   {\larger
%     What is the size of the Power Set of a \structure{finite} set?
%
%     \vfill
%
%     \begin{itemize}
%     \item Make a bijection between the power set and the binary string
%     \item Calculate the size of a binary string
%     \item Establish equality
%     \end{itemize}
%
%   }
% \end{frame}
