\documentclass{beamer}

\usepackage{amssymb,amsmath}
\usepackage{graphicx}
\usepackage{url}
\usepackage{color}
\usepackage{pagenote}[continuous,page]
\usepackage{relsize}		% For \smaller
\usepackage{url}			% For \url
\usepackage{epstopdf}	% Included EPS files automatically converted to PDF to include with pdflatex

%For MindMaps
% \usepackage{tikz}%
% \usetikzlibrary{mindmap,trees,arrows}%

%%% Color Definitions %%%%%%%%%%%%%%%%%%%%%%%%%%%%%%%%%%%%%%%%%%%%%%%%%%%%%%%%%
%\definecolor{bordercol}{RGB}{40,40,40}
%\definecolor{headercol1}{RGB}{186,215,230}
%\definecolor{headercol2}{RGB}{80,80,80}
%\definecolor{headerfontcol}{RGB}{0,0,0}
%\definecolor{boxcolor}{RGB}{186,215,230}

%%% Save space in lists. Use this after the opening of the list %%%%%%%%%%%%%%%%
%\newcommand{\compresslist}{
%	\setlength{\itemsep}{1pt}
%	\setlength{\parskip}{0pt}
%	\setlength{\parsep}{0pt}
%}

%\setbeameroption{show notes on top}

% You should run 'pdflatex' TWICE, because of TOC issues.

% Rename this file.  A common temptation for first-time slide makers
% is to name it something like ``my_talk.tex'' or
% ``john_doe_talk.tex'' or even ``discrete_math_seminar_talk.tex''.
% You really won't like any of these titles the second time you give a
% talk.  Try naming your tex file something more descriptive, like
% ``riemann_hypothesis_short_proof_talk.tex''.  Even better (in case
% you recycle 99% of a talk, but still want to change a little, and
% retain copies of each), how about
% ``riemann_hypothesis_short_proof_MIT-Colloquium.2000-01-01.tex''?

\mode<presentation>
{
  \usetheme{CambridgeUS}		% bem bacana - menu superior
  \usecolortheme{default}		% branco, azul clarinho
  \useoutertheme{default}
  \useinnertheme{circles}
  \setbeamercovered{invisible}
}

\beamertemplatenavigationsymbolsempty

%% Better looking blocks
\setbeamercolor{block title alerted}{use=structure,fg=black,bg=red!80!black}
\setbeamercolor{block body alerted}{use=structure,fg=black,bg=white!90!black}

\setbeamercolor{block title}{use=structure,fg=black,bg=blue!60!white}
\setbeamercolor{block body}{use=structure,fg=black,bg=white!90!black}

\usepackage[english]{babel}
\usepackage[latin1]{inputenc}
\usepackage{subfigure}

\usepackage{times}
\usepackage[T1]{fontenc}

%% makes the ppagenote command for figure references at the end.
\makepagenote
\renewcommand{\notenumintext}[1]{}
\newcommand{\ppagenote}[1]{\pagenote[Page \insertframenumber]{#1}}


\title[GB13604]{GB13604 - Maths for Computer Science}
\subtitle[]{Lecture 3 -- Number Theory}
\author[Claus Aranha]{Claus Aranha\\{\footnotesize caranha@cs.tsukuba.ac.jp}}
\institute[COINS]{College of Information Science}
\date[2018-10-17]{2018-10-17\\{\tiny Last updated \today}}

\begin{document}


\begin{frame}
  \maketitle

  \begin{center}
    {\smaller This course is based on Mathematics for Computer Science, Spring
    2015, by Albert Meyer and Adam Chlipala, Massachusetts Institute
    of Technology OpenCourseWare.}
    
    \includegraphics[width=0.2\textwidth]{../img/by-nc-sa}
  \end{center}
\end{frame}

\section{Introduction}

\subsection{In the last class}
\begin{frame}
  \frametitle{Summary Week 1 and 2}

  {\larger
  \begin{itemize}
  \item Proof by Cases
  \item Proof by Contradiction (Well Ordered Principle)
  \item Proof by Induction
  \item Sets Definition
  \item Sets Relationships
  \item Finite Set Sizes
  \end{itemize}
  }
\end{frame}

\begin{frame}
  \frametitle{Exercise Discussion}
\end{frame}

\begin{frame}
  \frametitle{For This Lecture...}

  {\larger
  
    Number Theory -- Textbook Chapter 8    
  
  \bigskip
  
  \begin{itemize}
  \item GCD and Extended GCD
  \item Modular Arithmetic, and Relatively Primes
  \item Euler's Theorem, and Rings
  \item RSA Algorithm
  \end{itemize}

  }
\end{frame}

\section{GCD}
\subsection{definitions}

\begin{frame}
  \frametitle{Some basic arithmetic assumptions}

  {\larger  
    For the proofs in this class, we can assume some default rules for
    arithmetic operators: *, +, -, ...

    \bigskip

    \begin{itemize}
    \item $a (b+c) = ab + ac$
    \item $ab = ba$
    \item $a(bc) = (ab)c$
    \item $a + 0 = a$
    \item $a - a = 0$
    \item $a + 1 > a$
    \item etc...
    \end{itemize}
    
  }
\end{frame}

\begin{frame}
  \frametitle{The Division Theorem}

  {\larger
    {\bf Axiom:}\\
    For any $b > 0$ and $a$ in $\mathbb{N}$, we have:

    \bigskip
    
    \begin{itemize}
    \item q = quotient(a,b)
    \item r = remainder(a,b)
    \end{itemize}

    \bigskip

    $\exists$ {\bf unique} $q$ and $r$ in $\mathbb{N}$ such as
    \begin{equation*}
      a = bq + r, 0 < r \leq a
    \end{equation*}

    \bigskip

    Take this by granted too!
  }
\end{frame}

\begin{frame}
  \frametitle{Divisibility}

  {\larger
    $c$ {\bf divides} $a (c|a)$ iff
    \begin{equation*}
      \exists k, a = k\times c.
    \end{equation*}

  \bigskip
  
  \begin{itemize}
  \item $5 | 15$ because $15 = 3 \times 5$
  \item $n | 0$ because $0 = 0 \times n$
  \item $1 | n$ because $n = n \times 1$
  \end{itemize}
  }
\end{frame}

\begin{frame}
  \frametitle{Simple Divisibility Facts}

  {\larger
    \begin{itemize}
    \item \structure{$c|a$ implies $c|(sa)$}\\
      \only<2->{$a = kc \text{ implies } (sa) =
      (sk)c$ \hfill multiply s on both sides}
    \item \structure{$c|a$ and $c|b$ implies $c | (a+b)$}\\
      \only<3->{$a = k_1c, b = k_2c, a+b = k_1c+k_2c = (k_1+k_2)c$}
    \item \structure{$c|a$ and $c|b$ implies $c | (sa+tb)$}\\
      \only<4->{$sa+tb$ is a \alert{linear combination} of a and b}

      \bigskip
      
      \only<5>{\alert{This one is pretty important!}}
      
    \end{itemize}
  }
\end{frame}

\begin{frame}
  \frametitle{Common Divisors}

  {\larger
    If $c|a$ and $c|b$, then \structure{c is a common divisor of a and b}.

    \bigskip

    {\bf Common divisors} of $a$ and $b$ also divide linear combinations of
    $a$ and $b$.
  }
\end{frame}

\subsection{GCD}
\begin{frame}
  \frametitle{Greatest Common Divisor}

  {\larger
    We define $gcd(a,b) ::=$ the greatest {\bf common divisor} of $a$ and $b$.

    \bigskip
    \begin{itemize}
    \item $gcd(10,12)$ = 2 \hfill ($10 = 2\times 5, 12 = 2\times 6$)
    \item $gcd(13,12)$ = 1 \hfill No common factors and $1|x, \forall x$
    \item $gcd(17,17)$ = 17
    \item $gcd(0,n)$ = n \hfill for $n > 0$
    \end{itemize}

    \bigskip

    Does {\bf one} gcd aways exists? (Yes, because of the Well
    Ordering Principle)
  }
\end{frame}

\begin{frame}
  \frametitle{Greatest Common Divisor}

  {\larger

    We define $gcd(a,b) ::=$ the greatest {\bf common divisor} of $a$ and $b$.

    \bigskip

    \begin{itemize}
    \item {\bf \structure{lemma}}: \structure{$p$ is prime} implies that \structure{gcd(p,a) = 1 or p};
    \item {\bf \structure{proof}}: The only divisors of $p$ are $\pm 1$ and $\pm p$.
    \end{itemize}
  }
  
\end{frame}

\subsection{Computing GCD}
\begin{frame}
  \frametitle{Euclidean Algorithm (GCD is easy to compute)}

  {\larger

    \structure{{\bf Remainder Lemma}}: gcd(a,b) = gcd(b, rem(a,b))
    \hfill for $b \neq 0$

    \bigskip
    
    \structure{{\bf Proof idea:}}
    \begin{itemize}
    \item $a = qb + r, 0\leq r < b$  \hfill (\structure{division axiom})
    \item Any divisor of two out of $\{a,qb,r\}$, must divide all three.\\
      (\alert{Check this yourself using slide 8})
    \item Therefore, $\forall m$ \structure{if} $m|a$ \structure{and} $m|b$
      \structure{then} $m|$rem(a,b)
    \end{itemize}
  }  
\end{frame}

\begin{frame}
  \frametitle{Example GCD (Using Remainder Lemma)}

  {\larger
    \begin{center}
      GCD(899, 493) -- a = 899, b = 493
    \end{center}

    \vfill

    \begin{itemize}
    \item $899 = 493 \times 1 + 406$ \hfill \structure{division axiom}
    \item GCD(899, 493) = GCD(493, 406) \hfill \structure{remainder lemma}
    \item<2-> GCD(493, 406) = GCD(406, 87) \hfill \structure{$493 =
      406\times 1 + 87$}
    \item<3-> GCD(406,87) = GCD(87,58) \hfill \structure{$406 = 87\times 4 + 58$}
    \item<4-> GCD(87,58) = GCD(58,29) = GCD(29,0) = \alert{29}
    \end{itemize}

    \bigskip

    This is a {\bf fast} algorith (proof later)
  }
\end{frame}

\begin{frame}
  \frametitle{GCD as a State Machine}

  {\larger

    \begin{itemize}
    \item \structure{States::=} $\mathbb{N}\times \mathbb{N}$ \bigskip
    \item \structure{Start State::=} $(a,b)$ \bigskip
    \item \structure{State Transitions::=}
      $(x,y) \rightarrow (y,rem(x,y))$ for $y\neq 0$ \bigskip
    \end{itemize}    
  }
\end{frame}

\begin{frame}
  \frametitle{GCD as a State Machine}

  {\larger
    {\bf \structure{Proof of Partial Correctness}}

    \vfill

    \begin{enumerate}
    \item \alert{We want to show}: P((x,y))::= [gcd(x,y) = gcd(a,b)]
    \item P(start) is \structure{trivially true}: (gcd(a,b) = gcd(a,b))
    \item P is a \structure{Preserved Invariant}:\\
      GCD(x,y) = GCD(y,rem(x,y))\hfill (remainder lemma)
    \item By 2 and 3, \structure{P holds for any state in the machine}.
      
      \bigskip
      
    \item So \structure{if the machine stops}, $x = gcd(a,b)$. Why?
      \begin{itemize}
      \item The machine only stops when $y = 0$
      \item GCD(x,0) = x
      \end{itemize}
    \end{enumerate}
    
  }
\end{frame}

\begin{frame}
  \frametitle{GCD as a State Machine}

  {\larger
    {\bf \structure{Proof of Termination}}

    \vfill
    
    \begin{itemize}
    \item At each transition, y is replaced with rem(x,y)
    \item $0 <$ rem(x,y) $\leq y$. \hfill (division axiom)
    \item So eventually $y = 0$, and \structure{the machine halts.}
      
      \bigskip

    \item \alert{how fast} does it halt?
    \item At each transition, x is replaced by y. Two cases:
      \begin{itemize}
      \item $y \leq x/2$ so x is halved this step.
      \item $y > x/2$ so rem(x,y) =
        x - y < (x/2), so x gets
        halved at the next step.
      \end{itemize}
    \item x gets halved (or even smaller) \alert{every two steps}.
    \item So number of steps is $\leq 2\log_2 b$
    \end{itemize}      
  }
\end{frame}

\subsection{Pulverizer}
\begin{frame}
  \frametitle{GCD and Linear Combinations}

  {\larger
  \begin{center}
    \structure{Extended Euclid Algorithm} or \alert{The Pulverizer}
  \end{center}

  \vfill

  {\bf Main Idea:}
  \begin{itemize}
  \item GCD(a,b) is a linear combination of a and b.
  \item GCD(a,b) $= sa + tb$.
  \item {\bf collorary:} All lin. comb. of a,b are multiples of GCD(a,b)

    \bigskip
    
  \item \alert{The Pulverizer} helps us find $s$ and $t$
  \end{itemize}
  
  }
\end{frame}

\begin{frame}
  \frametitle{The Pulverizer: Method}

  {\larger
    \structure{Calculate} euclid's algorithm:
    \begin{itemize}
    \item GCD(x,y) = GCD(y,rem(x,y)\hfill {\bf Start}: GCD(a,b)
    \end{itemize}

    \bigskip
    
    \structure{Keep track of} four coefficient: {\bf c,d,e,f}
    \begin{itemize}
    \item $x = ca+db$ and $y = ea+fb$
    \item {\bf at start:} x = 1a + 0b, y = 0a+1b
    \item {\bf update:} $x_{\text{next}} = y = ea + fb$
    \item $y_{\text{next}} =
      \text{rem}(x,y) = x - qy =
      ca+db-q(ea+fb)$
    \item $y_{\text{next}} = (c-qe)a+(d-qf)b$
    \end{itemize}
  }
\end{frame}

\begin{frame}
  \frametitle{The Pulverizer: Example}

  {\larger
    {\bf a = 899, b = 493}\\
    hfill (remember: $e_1 = c_0-q_0e_0, f_1 = d_0 - q_0f_0$) 

    \vfill
    
    \begin{tabular}{l|l|l|l|r|r|r|r}
      a & b & q & rem(a,b) & c & d & e & f\\
      \hline
      899 & 493 & 1 & 406 & 1 & 0 & 0 & 1\\
      493 & 406 & 1 & 87 & 0 & 1 & 1 & -1\\
      406 & 87 & 4 & 58 & 1 & -1 & -1 & 2\\
      87 & 58 & 1 & 29 & -1 & 2 & 5 & -9\\
      58 & 29 & 2 & 0 & 5 & -9 & -6 & 11\\
      29 & 0 & - & - & -6 & 11 & - & -\\      
    \end{tabular}

    \begin{center}
      GCD(899,493) = 29 = $-6\times 899 +11\times 493$
    \end{center}
  }  
\end{frame}

\begin{frame}
  \frametitle{The Pulverizer: One Weird Trick}

  {\larger

    \begin{equation*}
    \text{GCD}(899, 493) = -6\times 899 + 11\times 493
    \end{equation*}

    How can I get a positive coefficient for 899?

    \begin{equation*}
      \text{GCD}(899, 493) = (-6 + 493k)899 + (11 - 899k)493, \text{ for any }k
    \end{equation*}

    Let $k = 1$

    \begin{equation*}
      \text{GCD}(899, 493) = 487\times 899 - 888\times 493
    \end{equation*}
    
  }
\end{frame}

\subsection{Uses of GCD}

\begin{frame}
  \frametitle{Remember Robot 1.0?}

  {\larger

    \begin{itemize}
    \item It could move \structure{5 steps forward}, \alert{3 steps back}.
    \item How many moves it takes to reach ``8''?
    \item<2-> GCD(5,3) $= 1 = 2\times5 - 3\times 3$
    \item<3-> $8 = 8\times1 = (8\times2)5 - (8\times3) 3$
    \item<3-> 16 steps forward, 24 steps back.

      \bigskip

    \item <4-> Not the most efficient solution, but we can
      find any solution with this strategy.
    \end{itemize}
  }
\end{frame}

\begin{frame}
  \frametitle{Prime Factorization Theorem}

  {\larger             
  \begin{itemize}
  \item \structure{Lemma}: if p prime and p|ab, then p|a or p|b
  \item<2-> {\bf Proof}: suppose {\bf not(p|a)}, then GCD(p,a) = 1
  \item<3-> So: $\exists s,t. sa+tp = 1$, multiply everything by $b$
  \item<4-> sab + tbp = b
  \item<4-> $p|sab$ and $p|tbp$, so $p|(sab+tbp)$ and $p|b$ {\bf done.}

    \bigskip

  \item<5> \structure{Corolary}: if $p|a_1a_2\ldots a_m$ then $\exists i. p|a_i$
  \item<5> {\bf proof}: Induction on $m$    
  \end{itemize}

  }
\end{frame}

\begin{frame}
  \frametitle{Prime Factorization Theorem}

  {\larger
    \begin{block}{Fundamental Theorem of Arithmetic}
      Every Integer $> 1$ factors \alert{uniquely} into a
      weakly decreasing sequence of primes.
      \begin{equation*}
        n > 1, \hspace{1cm} n = p_1p_2p_3\ldots p_k, \hspace{1cm} p_1\geq p_2 \geq \ldots \geq p_k 
      \end{equation*}
    \end{block}

    \alert{Example} $61394323221 = 53\times37\times37\times37\times11\times11\times7\times3\times3\times3$
  }
\end{frame}

\begin{frame}
  \frametitle{Prime Factorization Theorem}
  
  {\larger
    {\bf Proof by Contradiction.}

    \begin{itemize}
    \item Suppose $n > 1$ does not have a unique
      prime factorization (it can be factored in two different ways).
    \item<2-> By WOP, there is a \structure{minimal} $n$ where theorem is false.
    \item<2-> $n = p_1p_2p_3\ldots p_k$ and $n = q_1q_2q_3\ldots q_{k'}$
    \item<3-> if $p_1 = q_1$ then we can cancel them, and $n$ is
      not smallest anymore. ($n' = p_2\ldots p_k = q_2\ldots q_{k'}$)
    \item<4-> {\bf So we assume } $q_1 > p_1$
    \item<4-> {\bf By the corolary} $q_1|n \rightarrow q_1|p_i \in p_1p_2\ldots p_k$
    \item<5> But, because $q_1 > p_i \forall i$, this is impossible. {\bf done.}
    \end{itemize}
  }
\end{frame}

\section{Congruences}
\begin{frame}
  \frametitle{}
  {\huge
    \begin{center}
      Congruences mod N
    \end{center}
  }
\end{frame}

\begin{frame}
  \frametitle{Congruence mod n: Definition}

  {\larger
    $a\equiv b$ (mod n) iff $n|(a-b)$

    \bigskip
    \structure{Examples:}
    \begin{itemize}
    \item <2->$30\equiv12$ (mod 9) \hfill because 9|(30-12)
    \item <3->$66666663 \equiv 788253$ (mod 10)
    \end{itemize}

    \vfill

    Congruence has many applications in crypto and hashing.

  }
\end{frame}

\begin{frame}
  \frametitle{Remainder Theorem}

  {\larger

    $a \equiv b$ (mod n) {\bf iff} rem(a,n) = rem(b,n)

    \bigskip
    
    \hfill (This is the CS \structure{``a\%n''} definition)

    \bigskip

    {\bf Proof:} \hfill(rem(a,b) = $r_{a,b}$)
    \begin{itemize}
    \item Let $a = q_an + r_{a,n}, \hspace{1cm} b = q_bn + r_{b,n}$
    \item {\bf if} $r_{a,n} = r_{b,n}$ then $a-b = (q_a-q_b)n \rightarrow n|(a-b)$
    \item {\bf also if} $n|(a-b)$ then $n|((q_a-q_b)n+(r_{a,n}-r_{b,n}))$
    \item \hspace{2cm} but $0 \leq r_{*,n} < n$ so $r_{a,n}-r_{b,n}$ must be 0
    \end{itemize}    
  }
\end{frame}

\begin{frame}
  \frametitle{Remainder Theorem: Consequences}

  {\larger
    $a \equiv b$ (mod n) means that rem(a,n) = rem(b,n).

    \bigskip
    \structure{Consequences}:

    \begin{itemize}
    \item $a \equiv b$ (mod n) {\bf implies that} $b \equiv a$ (mod n)
    \item $a \equiv b$ (mod n) {\bf and} $b \equiv c$ (mod n) {\bf implies}
      $a \equiv c$ (mod n)
    \item $a \equiv $rem(a,n) (mod n) \hfill (important!)
    \item {\bf If} $a \equiv b$ (mod n) {\bf then} $a+c \equiv b+c$ (mod n)
    \item {\bf If} $a \equiv b$ (mod n) {\bf then} $ac \equiv bc$ (mod n)
    \item {\bf If} $a \equiv b$ (mod n) {\bf and} $c\equiv d$ (mod n)\\
      {\bf then} $a+c \equiv b+d$ (mod n) {\bf and} $ac \equiv bd$ (mod n)
      
    \end{itemize}
  }
\end{frame}

\begin{frame}
  \frametitle{What does this mean?}

  {\larger
  \begin{center}
    Overall, arithmetic (mod n) is very similar to normal arithmetic.

    \bigskip

    If $a \equiv a'$ (mod n) and $a'$ is simpler,
    you can \alert{usually} replace in the formula to make it easier.

    \bigskip

    Using $a \equiv$ rem(a,n) (mod n) means that we can
    \structure{keep the numbers in modular arithmetic between 0 and n.}
  \end{center}
  }
\end{frame}

\begin{frame}
  \frametitle{Modular Arithmetic: Example}

  {\larger
    \begin{itemize}
    \item \structure{What is} $287^9 \equiv$ ? (mod 4)
    \item<2-> $287^9 \equiv 3^9$ (mod 4) {\bf because} $r_{287,4} = 3$
    \item<3-> $3^9 = ((3^2)^2)^2 \times 3$
    \item<3-> $((3^2)^2)^2\times 3 \equiv (1^2)^2 \times 3$ (mod 4)
      {\bf because} $9 \equiv 1$ (mod 4)
    \item<4-> $289^9 \equiv 3$ (mod 4)

      \vfill

    \item<4-> And we did not need to calculate any $x^9$!
      
    \end{itemize}
      
  }
\end{frame}

\subsection{Modular Inverses}

\begin{frame}
  \frametitle{Difference between Arithmetic and Modular Arithmetic}

  {\larger

    We saw that Arithmetic and Modular Arithmetic are similar \alert{but...}

    \bigskip

    \begin{itemize}
    \item $8\times2 \equiv 3\times2$ (mod 10)
    \item<2-> {\bf Can we do:} $8\times \not{2} \equiv 3\times \not{2}$ (mod 10)?
    \item<3-> $8 \not\equiv 3$ (mod 10)

      \bigskip

    \item<3-> \alert{We can't cancel arbitrarily!}
    \end{itemize}

    \begin{onlyenv}<4>
    \begin{block}{When can we cancel $ak \equiv bk$ (mod n)?}
      You can cancel when $k$ and $n$ have no common factors.

      \bigskip

      OR, when GCD(k,n) = 1      
    \end{block}
    \end{onlyenv}
  }  
\end{frame}

\begin{frame}
  \frametitle{Modular Inverses}

  {\larger

    \begin{itemize}
    \item \structure{Modular Inverse:} If GCD(k,n) = 1 then
      $\exists k', k\times k' \equiv 1$ (mod n)

      \bigskip
      
    \item If $ak \equiv bk$ (mod n), we can multiply both sides by $k'$
    \item $akk' \equiv bkk'$ (mod n) $\rightarrow 1a \equiv 1b$ (mod n)
    \end{itemize}

    \vfill

    $k$ \structure{has an inverse} (mod n) {\bf iff}
    $k$ \structure{is relatively prime} to $n$
  }
\end{frame}

\section{Euler Theorem}

\begin{frame}
  \frametitle{Euler's Function}

  {\larger

    Number of \structure{relatively primes} of $n$ between
    0 and $n$
    \begin{equation*}
      \Phi(n) ::= \# k\in[0,n), GCD(k,n) = 1
    \end{equation*}

    \bigskip

    Let us define:
    \begin{equation*}
      \text{gcd1}\{n\} ::= \{ k \in [0,n)|\text{GCD}(k,n) = 1\}
    \end{equation*}

    \begin{itemize}
    \item gcd1$\{7\} = \{1,2,3,4,5,6\}$ \hfill $\Phi(7) = 6$
    \item gcd1$\{12\} = \{1,5,7,11\}$ \hfill $\Phi(12) = 4$
    \end{itemize}
  }
\end{frame}

\begin{frame}
  \frametitle{Calculating $\Phi(n)$}

  {\larger
    \begin{itemize}
    \item If $n$ is prime, $\Phi(n) = n-1$
    \item If $n$ is a power of a prime, $\Phi(p^k) = p^k - p^{k-1}$
      \begin{itemize}
      \item Ex: $\Phi(9) = 3^2 - 3 = 6 \hspace{1cm} \{1,2,4,5,7,8\}$
      \end{itemize}
      
    \item If $n$ is $ab$ where GCD(a,b)=1, $\Phi(ab) = \Phi(a)\Phi(b)$
      \begin{itemize}
      \item Ex: $\Phi(12) = \Phi(3) \times \Phi(4) = (3-1)\times (2^2 - 2) = 4$
      \end{itemize}

      \vfill

    \item \alert{Euler's Theorem:} if GCD(k,n) = 1,
      $k^{\Phi(n)}\equiv 1$ (mod n)
      
    \end{itemize}    
  }
\end{frame}

\subsection{Euler's Proof and Z-Rings}
\begin{frame}
  \frametitle{The Ring of $\mathbb{Z}_n$}

  {\larger
    \begin{block}{Working with just Remainders}
      \begin{itemize}
      \item The integer interval $[0,n)$ under $+,\times
        (\mathbb{Z}_n)$ is called $\mathbb{Z}_n$.
        
        \bigskip
        
      \item $i+j (\mathbb{Z}_n) ::= \text{rem}(i+j,n)$
      \item $i\times j (\mathbb{Z}_n) ::= \text{rem}(i\times j,n)$
      \end{itemize}
    \end{block}

    \bigskip

    \begin{block}{Arithmetic in $\mathbb{Z}_n$}
      \begin{itemize}
      \item $3+6 = 2 (\mathbb{Z}_7)$
      \item $9\times8 = 6 (\mathbb{Z}_{11})$
      \item rem$(a,n)$ is equivalent to $r(a) (\mathbb{Z}_n)$
      \end{itemize}
    \end{block}
  }
\end{frame}

\begin{frame}
  \frametitle{$\equiv$ (mod n) and $\mathbb{Z}_n$}

  {\larger
    $i \equiv j$ (mod n) {\bf iff} $r(i) = r(j) (\mathbb{Z}_n)$

    \vfill

    As we saw before, most arithmetic rules apply to
    $\mathbb{Z}_n$ arithmetic.

    \bigskip

    \alert{No Cancelling Rule} -- Be careful that you cannot
    easily cancel multiplication!

    \begin{equation*}
      8\times \not{2} \neq 3\times\not{2} (\mathbb{Z}_{10})
    \end{equation*}
  }
\end{frame}

\begin{frame}
  \frametitle{$\mathbb{Z}^*_n$ -- Elements relatively prime to $n$}

  {\larger
    \begin{itemize}
    \item $i \in \mathbb{Z}^*_n$ iff gcd$(i,n) = 1$
    \item $i$ is cancellable in $\mathbb{Z}_n$
    \item $i$ has an inverse in $\mathbb{Z}_n$

      \bigskip

    \item $\Phi(n) ::= |\mathbb{Z}^*_n|$
    \item \structure{Euler's Theorem:} $k^{\Phi(n)} = 1 (\mathbb{Z}_n)$
      if $k \in \mathbb{Z}^*_n$
    \end{itemize}
  }
\end{frame}

%% Todo: Add proof of Euler's Theorem??

\section{RSA Algorithm}
\begin{frame}
  \frametitle{The RSA Encryption System}

  {\larger
    \begin{itemize}
    \item Public Key Cryptosystem;

      \bigskip
      
    \item<2-> \structure{Anyone} can send a secret (encrypted)
      message to the receiver \alert{without prior contact,
        using only public information}.

      \bigskip
      
    \item<3-> This sounds \structure{paradoxical}: How can someone
      construct a \alert{secret} message using only
      \structure{public} information?

    \end{itemize}
  }
\end{frame}

\begin{frame}
  \frametitle{RSA Cryptosystem: Basic Assumption}

  {\larger
    \begin{itemize}
    \item \structure{Basic Assumption:} {\bf One Way Functions}
      that are \structure{easy to compute} but \alert{hard to invert}

      \vfill

    \item It is \structure{easy} to compute the product $n$ of
      two \structure{large primes} $p$ and $q$ ($n = pq$)
      
      \bigskip

    \item It is \alert{very hard} to \structure{factor} $n$
      into $p$ and $q$.

    \end{itemize}
  }
\end{frame}

\begin{frame}
  \frametitle{RSA Cryptosystem: Preparations}

  {\larger
    \begin{itemize}
    \item \structure{sender} wants to send a message to \alert{receiver}
    \item \alert{rcv} generates primes $p,q$, $n::=pq$
    \item<2-> \alert{rcv} finds $e$ rel. prime to $(p-1)(q-1)$\\\hfill (hint: $(p-1)(q-1) = \Phi(n)$)
    \item<3-> \structure{(e,n)} ::= \structure{public key}.
      \alert{rcv} publishes it widely.
    \item<4-> \alert{rcv} finds $d::=e^{-1} (\mathbb{Z}^*_{(p-1)(q-1)})$
    \item<4-> $d$ ::= \structure{private key}, \alert{rcv} keeps it.
    \end{itemize}
  }
\end{frame}

\begin{frame}
  \frametitle{RSA Cryptosystem: Message}

  {\larger
    \begin{itemize}
    \item \structure{sender} encodes a message $m \in [1,n)$
    \item <2->\structure{sender} reads (e,n) and calculates
      $\hat{m} = m^e (\mathbb{Z}_n$)
    \item <2->\structure{sender} sends $\hat{m}$ to \alert{rcv}
    \item <3->\alert{rcv} calculates $\hat{m}^d = m (\mathbb{Z}_n)$

      \bigskip

    \item <4->\structure{Euler's Theorem} guarantees that $\hat{m}^d = m, d = e^{-1}, (\mathbb{Z}_n)$
    \end{itemize}
  }
\end{frame}

\begin{frame}
  \frametitle{RSA Cryptosystem: Requirements}

  {\larger
    \begin{itemize}
    \item Find two large primes, $p$ and $q$\\
      - Ok because: Lots of Primes\\
      - Need fast primality tester\\

      \bigskip
      
    \item Find $e$ relatively prime to $(p-1)(q-1)$\\
      - Ok because: Lots of relatively prime numbers\\
      - Fast because GCD$(e,(p-1)(q-1))$ is fast\\

      \bigskip
      
    \item Find $e^{-1} (\mathbb{Z}^*_{(p-1)(q-1)})$\\
      - Fast because of the Pulverizer

      \bigskip

    \item Check the book for the proofs.
    \end{itemize}    
  }
\end{frame}

\section{Conclusion}

\begin{frame}
  \frametitle{Summary of the Class}

  {\larger
    \begin{itemize}
    \item GCD algorithm (with proof) and Pulverizer

      \bigskip
      
    \item Arithmetic modulo n, and $\mathbb{Z}$ ring

      \bigskip

    \item Euler's Theorem
      
      \bigskip
      
    \item The RSA cryptosystem
    \end{itemize}
  }
\end{frame}

\begin{frame}
  \frametitle{Extra Reading}

  {\larger

    \begin{itemize}
    \item Proof for Euler's Theorem
      \vfill

    \item Relationship between SAT and factoring
    \end{itemize}
  }
  
\end{frame}


\end{document}
