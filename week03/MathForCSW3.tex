\documentclass[aspectratio=169]{beamer}

\usepackage{amssymb,amsmath}
\usepackage{graphicx}
\usepackage{url}
\usepackage{color}
\usepackage{pagenote}[continuous,page]
\usepackage{relsize}		% For \smaller
\usepackage{url}			% For \url
\usepackage{epstopdf}	% Included EPS files automatically converted to PDF to include with pdflatex

%For MindMaps
% \usepackage{tikz}%
% \usetikzlibrary{mindmap,trees,arrows}%

%%% Color Definitions %%%%%%%%%%%%%%%%%%%%%%%%%%%%%%%%%%%%%%%%%%%%%%%%%%%%%%%%%
%\definecolor{bordercol}{RGB}{40,40,40}
%\definecolor{headercol1}{RGB}{186,215,230}
%\definecolor{headercol2}{RGB}{80,80,80}
%\definecolor{headerfontcol}{RGB}{0,0,0}
%\definecolor{boxcolor}{RGB}{186,215,230}

%%% Save space in lists. Use this after the opening of the list %%%%%%%%%%%%%%%%
%\newcommand{\compresslist}{
%	\setlength{\itemsep}{1pt}
%	\setlength{\parskip}{0pt}
%	\setlength{\parsep}{0pt}
%}

%\setbeameroption{show notes on top}

% You should run 'pdflatex' TWICE, because of TOC issues.

% Rename this file.  A common temptation for first-time slide makers
% is to name it something like ``my_talk.tex'' or
% ``john_doe_talk.tex'' or even ``discrete_math_seminar_talk.tex''.
% You really won't like any of these titles the second time you give a
% talk.  Try naming your tex file something more descriptive, like
% ``riemann_hypothesis_short_proof_talk.tex''.  Even better (in case
% you recycle 99% of a talk, but still want to change a little, and
% retain copies of each), how about
% ``riemann_hypothesis_short_proof_MIT-Colloquium.2000-01-01.tex''?

\mode<presentation>
{
  \usetheme{CambridgeUS}		% bem bacana - menu superior
  \usecolortheme{default}		% branco, azul clarinho
  \useoutertheme{default}
  \useinnertheme{circles}
  \setbeamercovered{invisible}
}

\beamertemplatenavigationsymbolsempty

%% Better looking blocks
\setbeamercolor{block title alerted}{use=structure,fg=black,bg=red!80!black}
\setbeamercolor{block body alerted}{use=structure,fg=black,bg=white!90!black}

\setbeamercolor{block title}{use=structure,fg=black,bg=blue!60!white}
\setbeamercolor{block body}{use=structure,fg=black,bg=white!90!black}

\usepackage[english]{babel}
\usepackage[latin1]{inputenc}
\usepackage{subfigure}

\usepackage{times}
\usepackage[T1]{fontenc}

%% makes the ppagenote command for figure references at the end.
\makepagenote
\renewcommand{\notenumintext}[1]{}
\newcommand{\ppagenote}[1]{\pagenote[Page \insertframenumber]{#1}}


\title[GB13604]{GB13604 - Maths for Computer Science}
\subtitle[]{Lecture 3 -- Number Theory}
\author[Claus Aranha]{Claus Aranha\\{\footnotesize caranha@cs.tsukuba.ac.jp}}
\institute[COINS]{College of Information Science}
\date[2020-10-21]{2020-10-21\\{\tiny Last updated \today}}

\begin{document}


\begin{frame}
  \maketitle

  \begin{columns}
    \column{0.8\textwidth}
    {\smaller This course is based on Mathematics for Computer Science, Spring
    2015, by Albert Meyer and Adam Chlipala, Massachusetts Institute
    of Technology OpenCourseWare.}
    \column{0.2\textwidth}
    \includegraphics[width=\textwidth]{../img/by-nc-sa}
  \end{columns}
\end{frame}

\begin{frame}{Lecture Outline}

  Number Theory: From division to the RSA algorithm (textbook chapter 8).\bigskip

  \begin{itemize}
    \item Division and the Greatest Common Divisor (GCD);
    \item Primality, and simple cryptography;
    % First part of Turing crypto here
    \item Modular Arithmetic, and Euler's theorem;
    % Second part of Turing crypto here
    \item The RSA public key algorithm;
  \end{itemize}\bigskip

  Let's get started!
\end{frame}

\section{Divisibility}

\frame{
{Part 1: Divisibility}

\tableofcontents[currentsection,hideallsubsections, firstsection=1, sections={1-5}]
}

\subsection{Definitions}

\begin{frame}
  \frametitle{Talking about Division}

  How should we define the \emph{divide} operation on integers?\bigskip

  Consider $a/b$, for $a,b \in \mathbb{N}$ and $b > 0$. We have:
  \bigskip

  \begin{itemize}
  \item q = quotient(a,b)
  \item r = remainder(a,b)
  \end{itemize}

  \bigskip

  The {\bf Division Theorem} says that $\exists$ {\bf unique} $q$ and $r$ in $\mathbb{N}$ such as
  \begin{equation*}
    a = bq + r, 0 \leq r < b
  \end{equation*}\bigskip

  {\bf Example}: $16/3: a = 16, b = 3, q = 5, r = 1$;\hspace{1cm} $16 = 3\times 5 + 1$

\end{frame}

\begin{frame}
  \frametitle{Divisibility}

  We say that $c$ {\bf divides} $a (c|a)$ {\bf iff}
    \begin{equation*}
      \exists k \in \mathbb{N}, a = k\times c.
    \end{equation*}
    \begin{center}
      $c$ divides $a$ if there is a number $k$ where $a$ equals $c$ times $k$.
    \end{center}

  \bigskip

  \begin{itemize}
  \item $5 | 15$ because $15 = 3 \times 5$
  \item $n | 0$ because $0 = 0 \times n$ \hspace{1cm} every number divides 0
  \item $1 | n$ because $n = n \times 1$ \hspace{1cm} 1 divides every number
  \end{itemize}
\end{frame}

%%%%%%%%%%%%%%%%%%%

\begin{frame}{Implications of Divisibility (1/2)}

  \begin{block}{Lemma 1: $c|a \implies c|(sa)$}
    If $c$ divides $a$, then $c$ divides all multiples of $a$. {\bf Miniproof}: Multiply both sides by $s$:
    \begin{itemize}
      \item $a = kc$\hfill Definition of divisibility
      \item $sa = skc$\hfill Multiplies both sides by $s$
      \item $(sa) = (sk)\times c \implies c|sa$\qed
    \end{itemize}
  \end{block}

  \begin{block}{Lemma 2: $c|a \land c|b \implies c|(a+b)$}
    If $c$ divides $a$ and $b$, then $c$ divides $a+b$. {\bf Miniproof}: Distributive property of multiplication:
    \begin{itemize}
      \item $a = k_1c, b = k_2c$\hfill Definition of divisibility
      \item $a+b = k_1c + k_2c$\hfill Adding both equations together
      \item $(a+b) = (k_1+k_2)c \implies c|(a+b)$\qed
    \end{itemize}
  \end{block}
\end{frame}

\begin{frame}[t]{Implication of Divisibility (2/2)}

  \begin{block}{$c|a \land c|b \implies c | (sa+tb)$}
    If $c$ divides $a$ and $b$, then $c$ divides all \alert{linear combinations} of $a$ and $b$.\\
    {\bf This one is pretty important.}
  \end{block}
  Try to solve it by yourself first:\bigskip

  \begin{onlyenv}<2>
  \begin{itemize}
    \item Lemma 1: $c|a \implies c|ka$ for any $k$.\medskip
    \item Collorary: $c|a \land c|b \implies c|sa \land c|tb$\medskip
    \item Lemma 2: $c|a \land c|b \implies c|(a+b)$.\medskip
    \item Collorary: $c|sa \land c|tb \implies c|(sa+tb)$\qed
  \end{itemize}
  \end{onlyenv}
\end{frame}

\subsection{Common Divisors}

\begin{frame}
  \frametitle{Common Divisors}

    If $c|a$ and $c|b$, we say that c is a \structure{common divisor} of $a$ and $b$.\bigskip

    As we saw in the last slide, a common divisor of $a$ and $b$ will also divide the linear combinations of $a$ and $b$:
    \begin{equation*}
      c|a \land c|b \implies \forall s,t\in\mathbb{N}, c|(sa+tb)
    \end{equation*}\bigskip

    In the next section, let's talk in more details about $c$, $s$ and $t$.
\end{frame}

%%%%%%%%%%%%%%%%%%%%%%%%%%%%%%%%%%%%%%%%%%%%%%%%%%%%%%%%%%%%%%

\section{Greatest Common Divisor}

\frame{
{Part 2: Greatest Common Divisors}

\tableofcontents[currentsection,hideallsubsections, firstsection=1, sections={1-5}]
}

\begin{frame}{Greatest Common Divisor (GCD)}

  The \structure{Greatest Common Divisor} of $a$ and $b$ ($gcd(a,b)$), is the largest $c$ so that $c|a \land c|b$.\bigskip

  Examples:
  \begin{itemize}
  \item $gcd(10,12)$ = 2 \hfill $2|10 \land 2|12$
  \item $gcd(13,12)$ = 1 \hfill 13 and 12 have no common factors, and $1|x, \forall x$
  \item $gcd(17,17)$ = 17
  \item $gcd(0,n)$ = n \hfill $\forall n\in\mathbb{N}, n > 0, n|0$
  \end{itemize}
\end{frame}


\subsection{Computing the GCD}
\begin{frame}{Euclidean Algorithm}{The Remainder Lemma}

  The GCD is easy to compute, by using the \structure{Remainder Lemma}:\bigskip

  \begin{equation*}
    gcd(a,b) = gcd(b, \text{remainder}(a,b)), \text{ for } b\neq 0
  \end{equation*}\bigskip

  \begin{block}{Proof Idea}

    \begin{itemize}
    \item The division axiom states that: $a = qb + r, 0\leq r < b$;
    \item $c = gcd(a,b) \implies c|a \land c|qb$;
    \item $r = a - qb$, and $c|a \land c|qb \implies c|(a + (-1)*qb)$
    \item So $c|r$
    \end{itemize}
  \end{block}
\end{frame}

\begin{frame}{Euclidean Algorithm}{Calculating with the Remainder Lemma}

  To calculate the GCD of $a$ and $b$, we repeatedly calculate the remainder
  of $a$ and $b$, and replace $a$ with the remainder.\bigskip

  \begin{center}
    GCD(899, 493) -- a = 899, b = 493
  \end{center}\bigskip

  \begin{itemize}
  \item $899 = 493 \times 1 + 406$ \hfill \structure{division axiom}
  \item GCD(899, 493) = GCD(493, 406) \hfill \structure{remainder lemma}
  \item GCD(493, 406) = GCD(406, 87) \hfill \structure{$493 =
    406\times 1 + 87$}
  \item GCD(406,87) = GCD(87,58) \hfill \structure{$406 = 87\times 4 + 58$}
  \item GCD(87,58) = GCD(58,29) = GCD(29,0) = \alert{29}
  \end{itemize}
\end{frame}

\subsection{Proof of correctness}

\begin{frame}{Euclidean Algorithm}{State Machine}
  Let's use a State Machine to prove the correctness of Euclidean Algorithm:\bigskip

  \begin{itemize}
  \item \structure{States::=} $\mathbb{N}\times \mathbb{N}$\hspace{2cm} (values of $a$ and $b$)\bigskip
  \item \structure{Start State::=} $(a,b)$ \bigskip
  \item \structure{State Transitions::=}
    $(x,y) \rightarrow (y,rem(x,y))$ if $y\neq 0$ \bigskip
  \item \structure{End State::=} $y = 0$
  \end{itemize}\bigskip

  Remember that to prove correctness, we have to:
  \begin{itemize}
    \item Prove \structure{partial correctness}
    \item Prove \structure{termination}
  \end{itemize}
\end{frame}

\begin{frame}{Proof of GCD}{Proof of Partial Correctness}

  To prove partial correctness, we have to prove the \structure{preserved invariant} $P((x,y)) ::= [gcd(x,y) = gcd(a,b)]$.

  \begin{proof}
    We prove the preserved invariant $P((x,y))$ by induction:

    \begin{itemize}
      \item P(start) is trivial: $gcd(a,b) = gcd(a,b)$
      \item If $P((x,y))$ is true, then it is still true for any transition.
      \begin{itemize}
        \item There is only one transition: $(x,y) \to (y,\text{rem}(x,y))$
        \item The \structure{Remainder Lemma} says that: gcd(y,y) = gcd(y,rem(x,y));
      \end{itemize}
    \end{itemize}
    By these two items, we proved that $P()$ holds for any state in the machine.
  \end{proof}
\end{frame}

\begin{frame}{Proof of GCD}{Proof of Termination}
  \begin{itemize}
    \item The transition is $(x,y) \to (y,\text{rem}(x,y))$;
    \item By the division axiom, $0 \leq \text{ rem}(x,y) < y$;
    \item So $y$ becomes smaller after every transition;
    \item When $y = 0$, \structure{the machine halts};
  \end{itemize}\bigskip

  By the way, when the machine stops, the state is $(x_e,0)$. Because the preserved invariant state that $gcd(a,b) = gcd(x,y)$, and $gcd(x,0) = x$, the final result of the algorithm is that $gcd(a,b) = x_e$.
\end{frame}

\begin{frame}{How fast is the GCD?}

  Analysing the state machine, we can calculate how fast the GCD terminates:\bigskip

  \begin{itemize}
    \item At each transition, x is replaced by y. There are two cases:
      \begin{enumerate}
      \item $y \leq x/2$: so x is halved this step.\medskip
      \item $y > x/2$: so rem$(x,y) = x - y$ and x will get
        halved at the \emph{next} step.
      \end{enumerate}\bigskip
    \item So every two steps, x gets halved (or even smaller)
    \item This means that after $\leq 2\log_2x$ steps, the algorithm stops.
  \end{itemize}\bigskip

  So GCD$(a,b)$ is calculated after $2\log_2a$ steps.
\end{frame}


\subsection{The Pulverizer}
\begin{frame}{GCD and Linear Combinations}

  Remember that we showed that a divisor $c$ of $a$ and $b$ is also a divisor of $(sa+tb)$.\bigskip

  Note that $c$ is a divisor of itself too, so we can represent $c$ as $(sa+tb)$ for some $s$ and $t$.\bigskip

  The \structure{Extended Euclid Algorithm} (also called "The Pulverizer") can calculate $s$ and $t$ so that gcd$(a,b) = sa + tb$.\bigskip

  It is useful to note also that the gcd$(a,b)$ divides {\bf every linear combination} of $a$ and $b$.
\end{frame}

\begin{frame}{The Pulverizer Algorithm}

    \structure{Calculate} Euclid's algorithm as usual:
    \begin{itemize}
    \item GCD(x,y) = GCD(y,rem(x,y)\hspace{2cm} {\bf Start}: GCD(a,b)
    \end{itemize}

    \bigskip

    As we calculate GCD, \structure{keep track of} four coefficients: {\bf c,d,e,f}
    \begin{itemize}
    \item $x = ca+db$ and $y = ea+fb$
    \item {\bf at start:} $x = 1a + 0b$, $y = 0a+1b$
    \item {\bf update:} $x_{\text{next}} = y = ea + fb$
    \item $y_{\text{next}} =
      \text{rem}(x,y) = x - qy =
      ca+db-q(ea+fb)$
    \item $y_{\text{next}} = (c-qe)a+(d-qf)b$
    \end{itemize}
\end{frame}

\begin{frame}{The Pulverizer Algorithm}{Example}

    {\bf a = 899, b = 493}\\
    \hfill (remember: $e_1 = c_0-q_0e_0, f_1 = d_0 - q_0f_0$)

    \vfill

    \begin{tabular}{l|l|l|l|r|r|r|r}
      a & b & q & rem(a,b) & c & d & e & f\\
      \hline
      899 & 493 & 1 & 406 & 1 & 0 & 0 & 1\\
      493 & 406 & 1 & 87 & 0 & 1 & 1 & -1\\
      406 & 87 & 4 & 58 & 1 & -1 & -1 & 2\\
      87 & 58 & 1 & 29 & -1 & 2 & 5 & -9\\
      58 & 29 & 2 & 0 & 5 & -9 & -6 & 11\\
      29 & 0 & - & - & -6 & 11 & - & -\\
    \end{tabular}

    \begin{center}
      GCD(899,493) = 29 = $-6\times 899 +11\times 493$
    \end{center}
\end{frame}

\begin{frame}{The Pulverizer Algorithm}{Positive coefficients}

  So the Pulverizer calculates one linear combination of $a$ and $b$ which is equal to the GCD:

  \begin{equation*}
  \text{GCD}(899, 493) = -6\times 899 + 11\times 493
  \end{equation*}\bigskip

  It is possible to obtain other linear combinations, by defining $s$ and $t$ as follows:

  \begin{equation*}
    \text{GCD}(899, 493) = (-6 + 493k)899 + (11 - 899k)493, \text{ for any }k
  \end{equation*}\bigskip

  For example, if we set $k = 1$, we can find the following coefficients for $s$ and $t$:

  \begin{equation*}
    \text{GCD}(899, 493) = 487\times 899 - 888\times 493
  \end{equation*}

\end{frame}

\begin{frame}{The Pulverizer Algorithm}{Positive coefficients}

  Remember the robot from last class? The position of the robot was $x = 5a - 3b$. For any $x$, we can use the pulverizer to find $a$ and $b$.\bigskip

  \begin{itemize}
    \item Let's say that we want to find the path of the robot for $x = 8$
    \item gcd$(5,3) = 1 = 2\times5 - 3\times 3$, using the pulverizer.
    \item $8 = 8\times 1 = 8\times(2\times 5 - 3\times3) = (8\times2)5 - (8\times 3)3$
    \item Result: 16 moves forward, 24 moves back.
  \end{itemize}\bigskip

  This may not be the best solution for $x=8$, but it is an easy and fast algorithm to calculate a solution.
\end{frame}

\section{Primality}

% From prime numbers to simple criptography

\frame{
{Part 3: Primality}

\tableofcontents[currentsection,hideallsubsections, firstsection=1, sections={1-5}]
}

\begin{frame}{Prime Numbers}

  Prime numbers are numbers that have \structure{no divisors}, others than 1 and themselves.\bigskip

  In many senses, primes numbers are the {\bf building blocks} of arithmetic.\bigskip

  \begin{block}{The Fundamental Theorem of Arithmetic}
    Every positive integer has a {\bf unique} prime factorization.
  \end{block}
\end{frame}

\subsection{Prime Number Factorization}

% Fundamental theorem of arithmetic: Every number can be decomposed in primes

\begin{frame}{Proving Prime Number Factorization}{Step 1}
    {\bf Lemma 1:} If $p$ is prime and $p|ab$, then $p|a$ or $p|b$

    \begin{proof}
      \begin{itemize}
        \item Let's assume that $p$ {\bf does not} divide $a$. So gcd$(a,p) = 1$
        \item Consequently, $\exists s,t. sa+tp = 1$
        \item Multiply everything by $b: sab+tpb = b$
        \item $p|sab \land p|tpb \implies p|(sab+tpb) \implies p|b$
      \end{itemize}
    \end{proof}

    {\bf Corolary 1:} If $p|a_1a_2a_3\ldots a_m$, then $\exists 0 < i \leq m, p|a_i$\\
    \hfill (You can prove this by induction on m)
\end{frame}


\begin{frame}{Proving Prime Number Factorization}{Step 2}

  In lecture 1 we showed that every positive integer was a product of prime numbers.\\
  Now we need to show that the prime number factorization is {\bf unique}:

  \begin{proof}
    Proof by contradiction:
    \begin{itemize}
      \item By WOP, consider $n$ the smallest positive integer with a \alert{non-unique} factorization;
      \item $n = p_1p_2\ldots p_k$ and $n = q_1q_2\ldots q_k$
      \item Consider that these factors are {\bf ordered} ($q_1 \geq q_2 \geq \ldots$)
      \item $p_1 \neq q_1$ (else, we could cancel then out, and find a smaller $n$ with 2 factorizations)
      \item Assume $p_1 > q_1$. $p_1|n \implies p_1|q_1q_2q_3\ldots q_k$
      \item However, $p_1 > q_1 \geq q_2 \geq q_3 \ldots\geq q_k$, which is a contradiction.
    \end{itemize}

  \end{proof}
\end{frame}

\subsection{Using Primes}
% How many primes are out there? (Prime number theorem)
\begin{frame}{Prime Properties}
  Because prime numbers are so important, we are interested in knowing how many are there. Consider the $\pi(n)$ set of prime numbers between 2 and $n$:

  \begin{equation*}
    \pi(n) ::= [\{p\in[2..n]|p\text{ is prime}\}]
  \end{equation*}
  \bigskip

  The \structure{Prime Number Theorem} states how the size of $\pi(n)$ grows:
  \begin{equation*}
    \lim_{n\to\infty}\frac{\pi(n)}{n/\ln n} = 1
  \end{equation*}\bigskip

  As a general idea, the frequency of primes around an integer $n$ is $\frac{1}{\ln n}$. In other words, the quantity of primes gets smaller as $n$ gets bigger.
\end{frame}


% Few primes leads to difficulty of prime factorization

% Cryptography
  % Every number can be composed of primes
  % Prime factorization is difficult
\begin{frame}[fragile]{Prime Cryptography}{The Turing Algorithm}
  Alan Turing, among his many works in computer science, proposed a cryptography method using prime numbers.\bigskip

  \begin{enumerate}
    \item Transform the message you want to send ($m$) into a prime number:
\begin{verbatim}
  v  i  c  t  o  r  y       (not prime)         (prime)   **
  22 09 03 20 15 18 25 -> 22090320151825 -> 2209032015182513
\end{verbatim}
    \item Choose a secret key ($k$) as a large prime, and share it with the receiver;
    \item The \emph{sender} calculates the \emph{encrypted message} as follows, and sends it.
    \begin{equation*}
      \hat{m} = m\times k
    \end{equation*}
    \item The \emph{receiver} recovers the \emph{original message} using the secret key:
    \begin{equation*}
      m = \hat{m}/k
    \end{equation*}
  \end{enumerate}
\end{frame}

\begin{frame}{Prime Cryptography}{How secure is the Turing Algorithm?}
  Because of the difficulty of finding the prime factors of a large number, it is very hard to find $m$ and $k$ if you only have $\hat{m}$.\bigskip

  So if an adversary only knows $\hat{m}$, it is very hard to break this code. But what happens if the adversary has {\bf two} secret messages, $\hat{m_1}$ and $\hat{m_2}$?\bigskip

  Remember that $\hat{m_1} = k\times m_1$ and $\hat{m_2} = k\times m_2$. $k$ is a \structure{common divisor} of $\hat{m_1}$ and $\hat{m_2}$, so:
  \begin{equation*}
    k = gcd(\hat{m_1}, \hat{m_2})
  \end{equation*}\bigskip

  So it is very easy to find the secret key of this algorithm. Can we improve it?
\end{frame}

% Turing Code 1.0
\section{Modular Arithmetic}

% From prime numbers to simple criptography

\frame{
{Part 4: Modular Arithmetic}

\tableofcontents[currentsection,hideallsubsections, firstsection=1, sections={1-5}]
}

% Modular arithmetic and congruence

\subsection{Congruence}
\begin{frame}{Modular Arithmetic and Congruence}

  Using \structure{modular arithmetic}, we can make the Turing Algorithm stronger.\bigskip

  Modular arithmetic is centred around the concept of \structure{{\bf Congruence}}:

  \begin{equation*}
    a\equiv b (\text{modulo } n) \iff n|(a-b)
  \end{equation*}\bigskip

  \structure{Examples:}
  \begin{itemize}
\item $30\equiv12 (\text{modulo } 9)$,\hspace{4.3cm} because $9|(30-12)$
\item $66666663 \equiv 788253 (\text{modulo } 10)$,\hspace{2cm} because $10|66666663 - 788253$
  \end{itemize}
\end{frame}

\begin{frame}{Modular Arithmetic and Congruence}{Remainder Theorem}

  One important implication of the definition of equivalence is the remainder theorem:


  \begin{proof}
    Proof by two way implication. Let the remainder of $a$ and $b$ be $r_{a,b}$:

    \begin{itemize}
    \item Let $a = q_an + r_{a,n}$, and $b = q_bn + r_{b,n}$
    \item (right side) {\bf if} $r_{a,n} = r_{b,n}$ then $a-b = (q_a-q_b)n \implies n|(a-b)$
    \item (left side) {\bf if} $n|(a-b)$ then $n|((q_a-q_b)n+(r_{a,n}-r_{b,n}))$
    \begin{itemize}
      \item but $0 \leq r_{*,n} < n$ so $r_{a,n}-r_{b,n}$ must be 0
    \end{itemize}
    \end{itemize}
  \end{proof}
\end{frame}

% Congruence is preserved by addition and multiplication
\begin{frame}{Modular Arithmetic and Congruence}{Consequences of the Remainder Theorem}

  Remainder Theorem:
  \begin{equation*}
    a \equiv b (\text{mod } n) \iff \text{ rem}(a,n) = \text{ rem}(b,n)
  \end{equation*}

  \begin{itemize}
  \item $a \equiv b$ (mod n) {\bf and} $b \equiv c$ (mod n) {\bf implies}
    $a \equiv c$ (mod n)
  \item $a \equiv $ rem(a,n) (mod n) \hspace{1cm} (\alert{important!})\bigskip
  \item $a \equiv b$ (mod n) {\bf implies} $a+c \equiv b+c$ (mod n)
  \item $a \equiv b$ (mod n) {\bf implies} $ac \equiv bc$ (mod n)
  \item $a \equiv b$ (mod n) {\bf and} $c\equiv d$ (mod n)\\
    \hspace{2cm}{\bf implies} $a+c \equiv b+d$ (mod n) {\bf and} $ac \equiv bd$ (mod n)
  \end{itemize}\bigskip

  The last three consequences show that we can freely use \structure{addition} and \structure{multiplication} in modular arithmetic. Try to prove some of these consequences!
\end{frame}

% Remainder Arithmetic
\begin{frame}{Using Modular Arithmetic}{General Principle of Modular Arithmetic}

  \begin{block}{}
    You can simplify modulo operations composed of additions and multiplications by replacing integer operands by their remaiders.
  \end{block}

  Example: What is $287^9 \equiv$ ? (mod 4)

  \begin{itemize}
    \item Simplify $287^9$ (mod 4) to $3^9$ (mod 4), becase $r_{287,4} = 3$
    \item $3^9 \to 3^8\times 3 \to 9^4 \times 3$
    \item Simplify $9^4 \times 3$ (mod 4) to $1^4 \times 3$, because $r_{9,4} = 1$
    \item $289^9 \equiv 1^4 \times 3 \equiv 3$ (mod 4)
  \end{itemize}\bigskip

  We calculated a large exponent without actually calculating the exponentiation!
\end{frame}

% Remainder Inverse
% Remainder inverse and relatively prime numbers

\subsection{Modular Inverses}

\begin{frame}{Modular Arithmetic and Division}

  We so that modular arithmetic works for addition and multiplication, so what about \structure{division}?\bigskip

  \begin{itemize}
    \item $8\times 2 \equiv 3\times 2$ (mod 10)
    \item $8\times \not{2} \equiv 3\times \not{2}$ (mod 10)
    \item $8 \equiv 3$ (mod 10) \hspace{2cm} \alert{\bf FALSE!}
  \end{itemize}\bigskip

  We cannot cancel out multiplications arbitrarily!
\end{frame}

\begin{frame}{Modular Arithmetic and Division}{Modular Inverses}

  When we cancel the multiplication of a real number, it is equivalent of multiplying it by its {\bf multiplicative inverse}:

  \begin{equation*}
    an = bn \to a\not{n} = b\not{n} \to a(n\times \frac{1}{n}) = b(n\times \frac{1}{n})
  \end{equation*}

  So a multiplicative inverse of $n$ is the number $k$ so that $n\times k =1$.\bigskip

  In modulo arithmetic, we have: $n\times k \equiv 1$ (mod m). But, it turns out that it is not possible to find a $k$ that solves this equation for every $n$ and $m$!
\end{frame}


\begin{frame}{Modular Inverses}{When does $n$ have an inverse, and how can we find it?}

  It turns out that $n$ only has a modular inverse (module $m$) if $gcd(n,m) = 1$.\bigskip

  It is relatively easy to find the modular inverse:
  \begin{itemize}
    \item First we calculate gcd(n,m) using the pulverizer, and obtaining $s$ and $t$ so that $sm+tn=1$
    \item The multiplicative inverse of $n$ modulo $m$ is given by: $r_{m,t}$
    \item So $n\times r_{m,t} \equiv 1$ (mod $m$)
  \end{itemize}
\end{frame}

\subsection{Turing Code 2.0}

\begin{frame}{Turing Code 2.0}
  Using Modular Arithmetic, we can solve some of the problems of the cryptography algorithm described in the last section:

  \begin{enumerate}
    \item The sender and the receiver agree on a large prime number $n$, which will be the module, as well as the secret key $k < n$
    \item {\bf Encription:} The message $m$ is a prime number ($0 < m < n$). The sender calculates the secret message $\hat{m}$ using modulo multiplication:
    \begin{equation*}
      \hat{m} = m\times k \text{ mod }n
    \end{equation*}
    \item {\bf Decription:} The receiver multiplies the encrypted message $\hat{m}$ by the inverse of the key: $k^{-1}$ (mod $n$):
    \begin{equation*}
      m = \hat{m}0\times k^{-1} \text{ mod }n
    \end{equation*}
  \end{enumerate}
  We cannot find $k$ anymore by having two encrypted messages, so is this algorithm safe?
\end{frame}

\begin{frame}{Turing Code 2.0}{Breaking the Turing Code with a Plaintext Attack}

  The new algorithm can be broken by what is called a \structure{{\bf Plaintext Attack}}.\bigskip

  Imagine that an attacker is able to acquire an encrypted message $\hat{m}$ and its respective clear message $m$. To break this code:\bigskip

  \begin{itemize}
    \item Calculate the multiplicative inverse of $m$ modulo $n$: $m^{-1}$, using the Pulverizer.
    \item Multiply the result by the encrypted message:
    \begin{itemize}
      \item $m^{-1}\times \hat{m} = m^{-1}\times (m \times k) = (m^{-1} \times m) \times k = 1\times k$ (mod $n$)
    \end{itemize}
    \item This allows us to recover the secret key $k$.
  \end{itemize}\bigskip

  So this algorithm is not so good after all! :-(
\end{frame}

% Turing Code 2.0
\section{RSA Algorithm}

\frame{
{Part 5: RSA Algorithm}

\tableofcontents[currentsection,hideallsubsections, firstsection=1, sections={1-5}]
}


\begin{frame}{The RSA Cryptosystem}
  Turing's cryptography algorithm that we described in the previous sections was flawed. But there is a cryptographic algorithm based on modular arithmetic which is the base of much of the security of the internet.\bigskip

  The RSA algorithm involves computing the remainders of numbers raised to large powers.
\end{frame}

\subsection{Euler's Theorem}

\begin{frame}{Euler's Function}

  The number of integers between $0$ and $n$ that are relatively prime to $n$ is represented by \structure{Euler's Function}:
  \begin{equation*}
    \Phi(n) ::= \# k\in[0,n), GCD(k,n) = 1
  \end{equation*}
  \bigskip

  For example:
  \begin{itemize}
    \item $\Phi(7) = 6$, because $\{1,2,3,4,5,6\}$ are relatively prime to 7;
    \item $\Phi(12) = 4$, because $\{1,5,7,11\}$ are relatively prime to 12;
  \end{itemize}\bigskip
\end{frame}

\begin{frame}{Euler's Theorem}
  The value of $\Phi$ is important because of \structure{Euler's Theorem}: If $n$ and $k$ are relatively prime, then
  \begin{equation*}
    k^{\Phi(n)} \equiv 1 (\text{mod } n)
  \end{equation*}\bigskip

  Additionally, Euler's function and Euler's theorem gives us another way to calculate modular inverses:
  \begin{itemize}
    \item $k^{\Phi(n)} \equiv 1$ (mod $n$)
    \item $k^{\Phi(n)-1}\times k \equiv 1$ (mod $n$)
  \end{itemize}
  So $k$ and $k^{\Phi(n)-1}$ are inverses, modulo $n$.
\end{frame}

%% TODO: Prove Euler's Theorem (Page 274)


\begin{frame}{Calculating $\Phi(n)$}

  We can calculate $\Phi(n)$ quickly with some simple rules:\bigskip

  \begin{itemize}
  \item If $n$ is prime, $\Phi(n) = n-1$\bigskip

  \item If $n$ is a power of a prime, $\Phi(p^k) = p^k - p^{k-1}$
    \begin{itemize}
    \item Ex: $\Phi(9) = 3^2 - 3 = 6 \hspace{1cm} \{1,2,4,5,7,8\}$
    \end{itemize}\bigskip

  \item If $n$ is $ab$ where GCD(a,b)=1, $\Phi(ab) = \Phi(a)\Phi(b)$
    \begin{itemize}
    \item Ex: $\Phi(12) = \Phi(3) \times \Phi(4) = (3-1)\times (2^2 - 2) = 4$
    \end{itemize}
  \end{itemize}
\end{frame}

% RSA

\subsection{RSA Algorithm}
\begin{frame}
  \frametitle{The RSA Encryption System}

  The RSA, created in 1977, is a \structure{Public Key Cyrptosystem}. This means that, unlike the Turing algorithm, it is not necessary to exchange a secret key between the \emph{sender} and the \emph{receiver}.\bigskip

  The RSA uses Euler's theorem for encrypting and decrypting messages, and its security is based on the difficulty of factoring large numbers.
\end{frame}

\begin{frame}{The RSA System}{Algorithm}

  {\bf Preparation}
  \begin{itemize}
    \item Generate two distinct large primes, $p$ and $q$, which are secret
    \item Let $n = pq$
    \item Create a \structure{public key} $e \in [0..n)$ such that gcd$(e, (p-1)(q-1)) = 1$;
    \item Create a \structure{private key} $d \in [0..n)$ be the modular inverse of $e$ (mod $(p-1)(q-1)$);
  \end{itemize}\bigskip

  {\bf Encoding a Secret Message}
  \begin{itemize}
    \item Given a message $m \in [0..n)$, construct the encrypted message as $\hat{m} = m^e (\text{mod }n)$
  \end{itemize}\bigskip

  {\bf Decoding a Secret Message}
  \begin{itemize}
    \item Given an encrypted message $\hat{m}$, it is decrypted as $m = \hat{m}^d (\text{mod }n)$
  \end{itemize}\bigskip
\end{frame}

\begin{frame}{The RSA System}{Assumptions}

  \begin{itemize}
  \item \structure{Basic Assumption:} {\bf One Way Functions}
    that are \structure{easy to compute} but \alert{hard to invert}

    \bigskip

  \item It is \structure{easy} to compute the product $n$ of
    two \structure{large primes} $p$ and $q$ ($n = pq$)

    \bigskip

  \item It is \alert{very hard} to \structure{factor} $n$
    into $p$ and $q$.

  \end{itemize}
\end{frame}

%%%%%%%%%%%%%%%%%%%%%%%%%%%%%%%%%%%



\section{Conclusion}

\begin{frame}{Slide Credits}
  These slides were made by Claus Aranha, 2020. You are welcome to copy, re-use and modify this material, following the CC-SA-NC license.
  \bigskip

  These slides are based on "Mathematics For Computer Science (Spring 2015)", by Albert Meyer and Adam Chlipala, MIT OpenCourseWare. \url{https://ocw.mit.edu}.
  \bigskip

  Individual images in some slides might have been made by other
  authors. Please see the following slides for information about these cases.
\end{frame}

\begin{frame}[allowframebreaks]{Image Credits}
  \printnotes
\end{frame}

\end{document}
