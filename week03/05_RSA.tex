\section{RSA Algorithm}

\frame{
{Part 5: RSA Algorithm}

\tableofcontents[currentsection,hideallsubsections, firstsection=1, sections={1-5}]
}


\begin{frame}{The RSA Cryptosystem}
  Turing's cryptography algorithm that we described in the previous sections was flawed. But there is a cryptographic algorithm based on modular arithmetic which is the base of much of the security of the internet.\bigskip

  The RSA algorithm involves computing the remainders of numbers raised to large powers.
\end{frame}

\subsection{Euler's Theorem}

\begin{frame}{Euler's Function}

  The number of integers between $0$ and $n$ that are relatively prime to $n$ is represented by \structure{Euler's Function}:
  \begin{equation*}
    \Phi(n) ::= \# k\in[0,n), GCD(k,n) = 1
  \end{equation*}
  \bigskip

  For example:
  \begin{itemize}
    \item $\Phi(7) = 6$, because $\{1,2,3,4,5,6\}$ are relatively prime to 7;
    \item $\Phi(12) = 4$, because $\{1,5,7,11\}$ are relatively prime to 12;
  \end{itemize}\bigskip
\end{frame}

\begin{frame}{Euler's Theorem}
  The value of $\Phi$ is important because of \structure{Euler's Theorem}: If $n$ and $k$ are relatively prime, then
  \begin{equation*}
    k^{\Phi(n)} \equiv 1 (\text{mod } n)
  \end{equation*}\bigskip

  Additionally, Euler's function and Euler's theorem gives us another way to calculate modular inverses:
  \begin{itemize}
    \item $k^{\Phi(n)} \equiv 1$ (mod $n$)
    \item $k^{\Phi(n)-1}\times k \equiv 1$ (mod $n$)
  \end{itemize}
  So $k$ and $k^{\Phi(n)-1}$ are inverses, modulo $n$.
\end{frame}

%% TODO: Prove Euler's Theorem (Page 274)


\begin{frame}{Calculating $\Phi(n)$}

  We can calculate $\Phi(n)$ quickly with some simple rules:\bigskip

  \begin{itemize}
  \item If $n$ is prime, $\Phi(n) = n-1$\bigskip

  \item If $n$ is a power of a prime, $\Phi(p^k) = p^k - p^{k-1}$
    \begin{itemize}
    \item Ex: $\Phi(9) = 3^2 - 3 = 6 \hspace{1cm} \{1,2,4,5,7,8\}$
    \end{itemize}\bigskip

  \item If $n$ is $ab$ where GCD(a,b)=1, $\Phi(ab) = \Phi(a)\Phi(b)$
    \begin{itemize}
    \item Ex: $\Phi(12) = \Phi(3) \times \Phi(4) = (3-1)\times (2^2 - 2) = 4$
    \end{itemize}
  \end{itemize}
\end{frame}

% RSA

\subsection{RSA Algorithm}
\begin{frame}
  \frametitle{The RSA Encryption System}

  The RSA, created in 1977, is a \structure{Public Key Cyrptosystem}. This means that, unlike the Turing algorithm, it is not necessary to exchange a secret key between the \emph{sender} and the \emph{receiver}.\bigskip

  The RSA uses Euler's theorem for encrypting and decrypting messages, and its security is based on the difficulty of factoring large numbers.
\end{frame}

\begin{frame}{The RSA System}{Algorithm}

  {\bf Preparation}
  \begin{itemize}
    \item Generate two distinct large primes, $p$ and $q$, which are secret
    \item Let $n = pq$
    \item Create a \structure{public key} $e \in [0..n)$ such that gcd$(e, (p-1)(q-1)) = 1$;
    \item Create a \structure{private key} $d \in [0..n)$ be the modular inverse of $e$ (mod $(p-1)(q-1)$);
  \end{itemize}\bigskip

  {\bf Encoding a Secret Message}
  \begin{itemize}
    \item Given a message $m \in [0..n)$, construct the encrypted message as $\hat{m} = m^e (\text{mod }n)$
  \end{itemize}\bigskip

  {\bf Decoding a Secret Message}
  \begin{itemize}
    \item Given an encrypted message $\hat{m}$, it is decrypted as $m = \hat{m}^d (\text{mod }n)$
  \end{itemize}\bigskip
\end{frame}

\begin{frame}{The RSA System}{Assumptions}

  \begin{itemize}
  \item \structure{Basic Assumption:} {\bf One Way Functions}
    that are \structure{easy to compute} but \alert{hard to invert}

    \bigskip

  \item It is \structure{easy} to compute the product $n$ of
    two \structure{large primes} $p$ and $q$ ($n = pq$)

    \bigskip

  \item It is \alert{very hard} to \structure{factor} $n$
    into $p$ and $q$.

  \end{itemize}
\end{frame}
