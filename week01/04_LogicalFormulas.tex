\section{Logical Formulas}

\frame{\tableofcontents[currentsection,hideallsubsections, firstsection=2, sections={2-4}]}

% - Logical formulas
%    - Logical formulas can be expressed as truth tables.

\subsection{Predicate Cauculus and Truth Tables}
\begin{frame}
  \frametitle{Why Mathematical Language?}
  Human language can be imprecise, so we have mathematical language that can be more specific:\bigskip

  \begin{center}
    "Go to the supermarket to buy 1 milk pack. If they have eggs, buy 12."
  \end{center}
  \bigskip

  Which of the following is correct?
  \begin{itemize}
    \item If the supermarket has eggs, buy 1 milk pack and 12 eggs.
    \item If the supermarket has eggs, buy 12 milk packs.
  \end{itemize}
  \bigskip

  To avoid this imprecision, we prefer to use mathematical language when talking about logical relationships and proofs.
\end{frame}

% Truth table.

\begin{frame}{Predicate Calculus and Logical Operators}
  The mathematical language that we use in this lecture is called \emph{Predicate Calculus}. Predicate calculus connects {\bf Predicates} and {\bf Propositions} using logical operators.\bigskip

  Many of the logical operators you already know from boolean logic:
  \begin{itemize}
    \item {\bf AND}, {\bf OR}, {\bf XOR}, {\bf NOT}, etc...
  \end{itemize}\bigskip

  There are a few more unusual logical operators too:
  \begin{itemize}
    \item {\bf IMPLIES}, {\bf IFF}, {\bf FOR ALL}, {\bf EXISTS}
  \end{itemize}
\end{frame}

\begin{frame}{Predicate Calculus and Truth Tables}
  To evaluate a formula in predicate calculus, we can use \structure{Truth Tables}, which describe every possible truth value to each proposition.\bigskip

  {\bf Example:} P AND Q IMPLIES R\bigskip

  \begin{tabular}{|lll|l|l|}
    P & Q & R & P AND Q & P AND Q IMPLIES R\\
    \hline
    {\bf TRUE} & {\bf TRUE} & {\bf TRUE} & {\bf TRUE} & {\bf TRUE} \\
    {\bf TRUE} & {\bf TRUE} & FALSE & {\bf TRUE} & FALSE \\
    {\bf TRUE} & FALSE & {\bf TRUE} & FALSE & {\bf TRUE} \\
    {\bf TRUE} & FALSE & FALSE & FALSE & {\bf TRUE} \\
    FALSE & {\bf TRUE} & {\bf TRUE} & FALSE & {\bf TRUE} \\
    FALSE & {\bf TRUE} & FALSE & FALSE & {\bf TRUE} \\
    FALSE & FALSE & {\bf TRUE} & FALSE & {\bf TRUE} \\
    FALSE & FALSE & FALSE & FALSE & {\bf TRUE} \\
  \end{tabular}
\end{frame}


\subsection{Validity and Satisfiability}
% - In mathematics and proofs: Every/Exists and their relationship
%    with validity and satisfiability. (Not only boolean)

\begin{frame}{Logic Operators: "For All" and "Exists"}
  Two of the operators we mentioned are special, and deserve some special attention:

  \begin{block}{Operator: For all}
    For a predicate $P(x)$, FOR ALL $P(x)$ is True if $P(x)$ is true for {\bf every} value of x. It is equivalent to a chain of "AND"s:

    \[
      F(P(x)): \forall x_i \in X, P(x_0) \land P(x_1) \land \ldots \land P(x_n)
    \]

  \end{block}

  \begin{block}{Operator: Exists}
    For a predicate $P(x)$, EXISTS $P(x)$ is True if $P(x)$ is true for {\bf any} value of x. It is equivalent to a chain of "OR"s:

    \[
      E(P(x)): \exists x_i \in X, P(x_0) \lor P(x_1) \lor \ldots \lor P(x_n)
    \]
  \end{block}
\end{frame}

% - Validity: Always True
% - Satisfiability: True for some values
%    - Can test both with truth tables

\begin{frame}{Validity and Satisfiability}
  The logical operators "exists" and "for all" are closed linked to the concepts of "Validity" and "Satisfiability":\bigskip

  \begin{itemize}
    \item {\bf A logical formula is \structure{Valid} if}: The formula evaluates for true for every possible assignment of every variable.\medskip

    For example: $P \land \text{ NOT } P \implies Q$ is valid.\bigskip

    \item {\bf A logical formula is \structure{Satisfiable} if}: The formula evaluates for true for at least one possible assignment of variables.\medskip

    For example: $P \lor (Q \land R)$ is satisfiable
  \end{itemize}\bigskip
\end{frame}

% \begin{frame}{Validity and Satisfiability}{Proofs and Validity}
%   There are many important implications and uses for the concepts of validity and satisfiability. For example, we can use these concepts when designing proofs.\bigskip

%   If we define a proposition that we want to prove as a logical formula, we can say that the proposition is true if the logical formula is \structure{Valid}.\bigskip

%   On the other hand, we can define a \structure{proof by contradiction} by showing that a logical formula that indicates the negative of the proposition is \structure{satisfiable}.
% \end{frame}

% - Equivalence: A and B have exactly the same truth table (IFF)
%    - Rewriting code

\begin{frame}{Equivalence}{Comparison of Two Formulas}
  Another related concept is \structure{Equivalence}. We say that two logical formulas are equivalent, if their result is identical for every variable assignment.\vfill

  For example: $\text{NOT }(P \lor Q)$ is equivalent to $\text{NOT } P \land \text{ NOT } Q$\\
  \hfill (DeMorgan's Law)\vfill

  The equivalence of two formulas is useful when rewriting code, and showing that two different pieces of code have the same result.
\end{frame}

\begin{frame}[t]{Verifying Validity, Satisfiability, and Equivalence}
  How can we verify that some logical formulas are (Valid | Satisfiable | Equivalent)?\bigskip 

  One way to do that is to examine the truth tables of the formulas:

  \begin{itemize}
    \item {\bf A formula is satisfiable}: At least one line in the truth table evaluates to TRUE.
    \item {\bf A formula A is valid}: The formula $\lnot{A}$ is NOT satisfiable.
    \item {\bf Two formulas (X,Y) are equivalent}: The formula $(A \land B) \lor (\lnot{A}\land\lnot{B})$ is valid.
  \end{itemize}\bigskip

  However, there is a problem: The size of the truth table is $2^n$, where $n$ is the number of variables!
\end{frame}


\subsection{The Satisfiability Problem (SAT problem)}

\begin{frame}{The Satisfiability Problem}

  Consider the problem of simplifying a computer program: Given a program defined as a logical formula $A$, we want to find a smaller formula $B$ that has the same functionality.\vfill

  We can test if a certain $B$ is equivalent to $A$ by testing if the expression $A \iff B$ is {\bf valid}. Alternatively, We can test that $B$ is {\bf not} equivalent to $A$ by testing if $\text{NOT }(A \iff B)$ is {\bf satisfiable}. If we can find only one variable assignment where $A$ and $B$ are not equal, then we can discard the program candidate $B$.\vfill

  This kind of analysis is useful for making programs run faster, or for creating simpler and cheaper hardware.
\end{frame}

\begin{frame}{The Satisfiability Problem}{Proving equivalences}
  The basic algorithm for proving equivalence in a SAT problem is to test each combination of variables (each line in the truth table). As we discussed before, the number of lines is $2^n$, so this can take a very long time.\vfill

  Interestingly, if we KNOW one set of variables that satisfy the formula, it is very quick to test it. Just evaluate the formula.\vfill

  This characteristic of SAT: "Very slow to find the answer, very fast to check the answer", is one of the key characteristics of NP-hardness. If you can find a quick solution to the SAT problem, you would become a very famous computer scientist!
\end{frame}
