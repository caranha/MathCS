\section{Logical Formulas}

\frame{\tableofcontents[currentsection,hideallsubsections, firstsection=2, sections={2-5}]}

\begin{frame}
  \begin{center}
    Propositions and Logic
  \end{center}
\end{frame}

\subsection{Ambiguous Language}
\begin{frame}
  \frametitle{Why Mathematical Language?}

  \begin{itemize}
  \item Greeks carry swords or javelins.

    \bigskip

  \item Greeks carry bronze or copper swords.
  \end{itemize}
\end{frame}

\begin{frame}
  \frametitle{Mathematical Language}
  \begin{itemize}
  \item Mathematical Language helps create non-ambiguous statements.
    \bigskip

  \item We will not through all Logic operators here.
    \bigskip

  \item However, it is important to understand that they are based on
    \structure{binary} or \structure{boolean} logic.
  \end{itemize}
\end{frame}

%% Question -- Should I add slides on binary mathematical operations?
%% How do I segue into this topic?

\subsection{Truth Tables}
\begin{frame}
  \frametitle{Mathematical Language / Binary Logic}

  Example: X \structure{XOR} Y

  \bigskip

  \begin{tabular}{ll|l}
    X & Y & X XOR Y\\
    \hline
    TRUE & TRUE & FALSE\\
    TRUE & FALSE & TRUE\\
    FALSE & TRUE & TRUE\\
    FALSE & FALSE & FALSE\\
  \end{tabular}

  \bigskip

  \begin{itemize}
  \item A \structure{Truth Table} is a way to understand a logic
    operator.
  \item We can use logic operators to transform \alert{ambiguous natural
    language sentences} into \structure{clear logical propositions.}
    \begin{itemize}
    \item Greeks carry bronze or copper swords.
    \item Greek carry bronze sword XOR greek carry copper sword.
    \end{itemize}
  \end{itemize}
\end{frame}

\begin{frame}
  \frametitle{Binary Logic and Truth Tables}

  The \structure{truth table} allows us to analyze a logical formula:

  \bigskip

  \begin{itemize}
  \item Is it always true? Is it always false?
  \item Is it equivalent to another logical formula?
  \end{itemize}

  To analyze a formula using the truth table, I need to analyse the
  value of each variable.
\end{frame}

\begin{frame}
  \frametitle{Evaluation of a Formula}

  Given the following variables:\\ P = \structure{True}, Q = \structure{True}, R = \alert{False}

  \vfill

  How do we \structure{evaluate} the following formula?
  \begin{center}
    NOT(NOT(P) OR Q) AND (R OR (P XOR Q))
  \end{center}
\end{frame}

\begin{frame}
  \frametitle{Comparison of Two Formulas}

  We can decide whether two logical formulas are
  \structure{equivalent} if the final column of their truth table is
  identical.

  \vfill

  For example, let's prove DeMorgan's Law:
  \begin{center}
    NOT(P OR Q) \structure{equiv to} NOT(P) AND NOT(Q)
  \end{center}
\end{frame}

\subsection{Satisfiability}

\begin{frame}
  \frametitle{Satisfiability and Validity}

  \begin{itemize}
  \item A logic formula is \structure{satisfiable} if it is true for
    \alert{at least one} assignment.
  \item A logic formula is \structure{valid} if it is true for
    \alert{all} assignments.
  \end{itemize}

  \vfill

  \begin{itemize}
  \item \structure{Satisfiable:} NOT(B)
  \item \structure{Not Satisfiable:} B AND NOT(B)
  \item \structure{Valid:} B OR NOT(B)
  \end{itemize}
\end{frame}

\begin{frame}
  \frametitle{Checking for Validity and Satisfiability}

  Checking if a logic formula is satisfiable or not is a
  \structure{very importan problem} in CS.

  \bigskip

  But how to do it?

  \bigskip

  \alert{Alert!} If you try to use a truth table, the size of the
  table grows with the number of variables:
  \begin{itemize}
  \item 1 variable - 2 lines
  \item 2 variables - 4 lines
  \item 10 variables - 1024 lines
  \item n variables - $2^n$ lines...
  \end{itemize}
\end{frame}

\begin{frame}
  \frametitle{Checking for Validity and Satisfiability}
  \begin{itemize}
  \item Is there an efficient way to test for satisfiability? (SAT)
    \bigskip

  \item The Efficient SAT problem is equivalent to the P=NP problem
    \bigskip

  \item The validity problem is also related to the SAT problem.
  \end{itemize}
\end{frame}

%\section{1.5 -- Quantifiers}
\begin{frame}
  \begin{center}
    Logic Quantifiers
  \end{center}

  \bigskip
  \begin{itemize}
  \item For all: $\forall$
  \item Exists: $\exists$
  \end{itemize}
\end{frame}

\subsection{Predicates and quantifiers}
\begin{frame}
  \frametitle{What is a Predicate?}

  A predicate is a proposition with variables in it:

  \begin{equation*}
    P(X,Y) ::= [X+2 = Y]
  \end{equation*}

  \vfill

  The truth value of a predicate depends on the values of the variables:
  \begin{itemize}
  \item $X = 1, Y = 3$, P(X,Y) is True
  \item $X = 2, Y = 2$, P(X,Y) is False
  \end{itemize}
\end{frame}

\begin{frame}
  \frametitle{Quantifiers}

  \begin{itemize}
  \item $\forall x$ -- For ALL X

    \bigskip

  \item $\exists y$ -- There exists SOME Y
  \end{itemize}

  \vfill

  $\forall x$ works like \structure{AND}. For example:
  \begin{equation*}
    \forall x, x \in \{1,2,3\}|P(X) \text{ \alert{equiv} } P(1) \text{ AND } P(2) \text{ AND } P(3)
  \end{equation*}

  \bigskip

  $\exists y$ works like \structure{OR}. For example:
  \begin{equation*}
    \forall x, x \in \{1,2,3\}|P(X) \text{ \alert{equiv} } P(1) \text{ OR } P(2) \text{ OR } P(3)
  \end{equation*}
\end{frame}

\begin{frame}
  \frametitle{Quantifiers Example}

  For $x,y \in \mathbb{N}$ (x and y range over the integers).

  \begin{equation*}
    Q(Y) ::= \exists x. x < y.
  \end{equation*}

  \begin{itemize}
  \item Q(3) is \structure{True}. ($[x < 3]$ is T for $x = 1$)
  \item Q(1) is \structure{True}. ($[x < 1]$ is T for $x = 0$)
  \item Q(0) is \alert{False}. ($[x < 0]$ is not T for any $x\in\mathbb{N}$)
  \end{itemize}

  \bigskip

  What about a simple example for $\forall$?
\end{frame}

\begin{frame}
  \frametitle{Ordering Quantifiers}
  What is the difference when we order $\exists$ and $\forall$?

  \bigskip

  \begin{block}{Example 1: Medicines}
    $\forall d \in$ diseases. $\exists m \in$ medicine.\\
    $m$ cures $d$
  \end{block}

  \begin{block}{Example 2: Panacea}
    $\exists m \in$ medicine. $\forall d \in$ diseases.\\
    $m$ cures $d$
  \end{block}

  We need to be careful when writing mathematical notation!

\end{frame}

\begin{frame}
  \frametitle{Validity and Predicates}
  \begin{itemize}
  \item Propositional Validity: A \structure{proposition} is true for
    all truth assignments of variables.
    \begin{itemize}
    \item Example: (P implies Q) OR (Q implies P)
    \end{itemize}

    \bigskip

  \item Predicate Calculus Validity: A \structure{predicate} is valid
    when it is true for all domains.
    \begin{itemize}
    \item Example: $\forall z. [P(z) \land Q(z)] \rightarrow [\forall x.P(x) \land \forall y.Q(y)]$
    \end{itemize}
  \end{itemize}


\end{frame}
