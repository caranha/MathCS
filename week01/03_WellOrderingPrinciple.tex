\section{Well Ordering Principle}

\frame{\tableofcontents[currentsection,hideallsubsections, firstsection=2, sections={2-5}]}

\subsection{Well Ordering Outline}

\begin{frame}
  \frametitle{The Well Ordering Principle}

  \begin{itemize}
  \item It is a very obvious (but very useful) principle in Mathematics;
    \bigskip

  \item It is so obvious that you have already used it without knowing;
  \end{itemize}

\end{frame}

\begin{frame}
  \frametitle{The Well Ordering Principle}

  {\Large
  \begin{center}
    Every non-empty set of\\
    \structure{Non-negative Integer Numbers}\\
    has one \alert{smallest element}
  \end{center}}
\end{frame}

\begin{frame}
  \frametitle{The Well Ordering Principle}
  \begin{center}
    Obvious? \structure{yes} \hspace{2cm} Trivial? \alert{no}
  \end{center}
  \vfill

  \begin{itemize}
  \item Every non-empty set of non-negative \structure{rational}
    numbers has one smallest element?

    \bigskip

  \item Every non-empty set of \structure{integers} numbers has one
    smallest element?
  \end{itemize}
\end{frame}

\begin{frame}
  \frametitle{Well Ordering Examples}
  {\large
    \begin{itemize}
    \item What is the smallest age of the U.Tsukuba students?
    \item What is the smallest number of cells in any animal?
    \item What is the smallest number of coins = 876 yens?
    \end{itemize}
  }
\end{frame}

\subsection{Sample Proof}
\begin{frame}
  \frametitle{Proof $\sqrt{2}$ is irrational using well ordering}

  \begin{itemize}
    \item if $\sqrt{2}$ is rational, then exist $m,n$ so that
      $\sqrt{2} = \frac{m}{n}$
    \item We can always find $m,n > 0$ such as they have no common
      factors.
    \item \alert{Why} always?
  \end{itemize}
\end{frame}

\begin{frame}
  \frametitle{Proof $\sqrt{2}$ is irrational using well ordering}
  \begin{itemize}
  \item Suppose that we choose the \alert{smallest} $m,n$.
    \bigskip

  \item Using the same idea as the previous proof, we show that both
    numbers must be divisible by two. ($m' = m/2, n' = n/2$)
    \bigskip

  \item Now we found a number \alert{smaller than the smallest! (contradiction!)}
  \end{itemize}
\end{frame}

\subsection{More Well Ordering Proofs}
\begin{frame}
  \frametitle{More Proofs Using the Well Ordering Principle}

  \begin{itemize}
  \item (Easy) Every integer $i > 1$ is a product of primes.

    \bigskip

  \item (Medium) Every number is \structure{Postal}.
    \begin{itemize}
    \item A number $n$ is postal if $n+8$ can be composed of a sum of
      ``threes'' and ``fives''
    \end{itemize}

    \bigskip

  \item (Difficult) $1 + r + r^2 + \ldots + r^n = \frac{r^n-1}{r-1}$
  \end{itemize}
\end{frame}

\begin{frame}
  \frametitle{General form for a Well Ordering proof}
  You want to prove that $\forall n \in \mathbb{N}, P(n)$ using WOP.

  \begin{enumerate}
  \item Define a set of counter examples $C$, $C ::=\{n \in
    \mathbb{N}|\text{ not } P(n)\}$
  \item Assume the minimum element of $C$ exists, $m$, by WOP
  \item Find a contradiction, for example:
    \begin{itemize}
    \item Find a contradiction $c \in C, c < m$;
    \item Show that $P(m)$ is actually True;
    \end{itemize}

  \end{enumerate}
\end{frame}

%\begin{frame}
%  \frametitle{Bogus WOP proof: Every Fibonacci Number is even}
%\end{frame}
